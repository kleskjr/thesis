\chapter{Introduction}


  \textit{ There is a story that Simonides was dining at the house of a wealthy
    nobleman named Scopas at Crannon in Thessaly, and chanted a lyric poem
    which he had composed in honor of his host, in which he followed the custom
    of the poets by including for decorative purposes a long passage referring
    to Castor and Pollux; whereupon Scopas with excessive meanness told him he
    would pay him half the fee agreed on for the poem, and if he liked he might
    apply for the balance to his sons of Tyndaraus, as they had gone halves in
  the panegyric.  }

  \textit{ The story runs that a little later a message was brought to
    Simonides to go outside, as two young men were standing at the door who
    earnestly requested him to come out; so he rose from his seat and went out,
    and could not see anybody; but in the interval of his absence the roof of
    the hall where Scopas was giving the banquet fell in, crushing Scopas
    himself and his relations underneath the ruins and killing them; and when
    their friends wanted to bury them but were altogether unable to know  them
    apart as they had been completely crushed, the story goes that Simonides
    was enabled by his  recollection of the place in which each of them had
    been reclining at table to identify them for separate interment; and that
    this circumstance suggested to him the \textbf{discovery of the truth that
    the best aid to clearness of memory consists in orderly arrangement}.  }

  \textit{ He inferred that persons desiring \textbf{to train this faculty must
    select localities and form mental images of the facts they wish to
    remember and store those images in the localities, with the result that
    the arrangement of the localities will preserve the order of the facts,
    and the images of the facts will designate the facts themselves}, and we
    shall employ the localities and images respectively as a wax writing tablet
  and the letters written on it.}\footnote{\cite{Cicero}, Book 2, 86.352--54.}

\section{Sequences as behaviour substrate: a historical overview}

  Simonides of Ceos was both a celebrated and condemned lyric poet in Ancient
  Greek world that lived in the period 556--468 BC. He has the fame of the
  inventor of 4 letters of the Greek alphabet ($\eta, \, \psi, \, \xi,\,{\rm
  and } \,\, \omega$) and is considered as the first commercial poet who
  created songs and odes for pay. Moreover, he is in the focus of the oldest
  reference (from 80 BC; \citep{Rhetorica}) for a mnemonic technique called
  `Method of Loci' or `Memory palace' (see the legend above from
  \cite{Cicero}). The method of loci has survived over the centuries as a
  technique to help orators, artists, clerks and encyclopedist to remember
  information of virtually any quality and quantity \citep{Yates66}. Nowadays,
  the method takes a central place in the training of every memory champion
  \citep{Foer2011}.
  
  How does this mnemonic technique work? Can a naive person actually remember a
  large list of items consisting of tens or even hundreds of elements? In its
  essence, the method of loci is a mnemonic technique based on images and
  places. To deploy the method, one needs a predefined trajectory in some very
  well known spatial environment (e.g., a good starting point for beginners is
  imagining home). Then, to imprint the list, each item is sequentially placed
  in the virtual trajectory (e.g., entrance door, telephone cupboard,
  bookshelf, door to the living room, sofa, sofa table, etc.) and imagined as
  vividly as possible. It is suggested to recruit as many sensory modalities as
  possible when imagining the placed objects as this creates more hooks for
  later revoke of the items. For the following recall one needs only to do a
  virtual walk through the same trajectory and to recollect the items from
  their places (loci). A vivid imagination is always beneficial for these
  mental exercises.

  %Laura's subject with super memory, having a condition of astasia? has been implementing a similar method, probably unconsciously.. (or a paragraph down?)
  %every super memory human tested so far has been using some memory technique that relies on methos of loci or story
  %creation. Even Luria's subject...colorful memory
  %sequences, tables that could be read by raws, columns and diagonals; after 16 years still remembers
  %using streets from his home town or moscow to place items that has be remembered when given very long lists
  %very strong imagination,  severe synestisia where he feels the color, shape, weight, touch, etc of every word he hears or reads..certain problems in concntration while reading and especially while eating..
  %He never changed items with similar items or with synonims, however, during recall often he could skip some item that was placed in a `dark place' or was not very well distinguished form the background..
  %weak in the logical `organisation of memorizing'
  %bad memory on faces: they are so changeable, changing colors all the time, depending on the mood of the person, on the moment of meet..

  %It is, however, an open question whether the sequential arrangement is an
  %universal quality, or whether some more complex schemata can be used for
  %memory formation and recall. For example, can a two-dimensional pattern be
  %used as a base for the storage of new memories?

  Since ancient times it is known that memory can greatly benefit from
  predefined sequences. All known mnemonic techniques such as the method of
  loci, the memory palace, and the story method rely on the sequential storage
  and the recall of associated memories.

  Sequences have been proposed also as a more general model for the occurrence
  and development of the mental processes. The oldest known references to the
  concept of association of thoughts and ideas can be tracked to Plato and
  Aristotle from the ancient world \citep{Plato:Phaedo, Bloch2007}. Later, in the western world the thought
  sequence concept was further developed by early modern philosophers
  philosophers \citep{Hobbes, Locke, Hume, Hume2, Stewart}. David Hartley,
  founder of associationist school of psychology in the 18th century proposed
  that even the most complex thought processes can be explain as sequential
  activations of clusters of elementary senses and representations. At the turn
  between 19th and 20th century, a lot of work devoted on the associations was
  contributed in the field of psychology. The thought experiment of William
  James' bear attempts to explain the physiological response of the body as
  sequence of mental events after the sensory arousal of seeing a bear
  \citep{James1884}. \cite{Pavlov1897} published his seminal work on the
  conditioned reflex, where he demonstrated the physiological effects of
  classical conditioning. Further work in experimental psychology conceived
  that sequences of experiences and their representations are in the base of
  motor action, cognition and judgment \citep{Watt1904, Titchener1905,
  Washburn1916}.

    %Cajal: neuron doctrine, neurons have 3 parts and propagate impulses only in
    %one direction from one neuron to other; bold assumtption from his side
    %based solely on anatomical observations... ref textura del sistema nerviso... and linas2003, Nature 

    %- from psychology(washburn watson) to neurophysiologistb (lashley, in The problem of serial order in behavior) talking about hierarchical organization of plans, and chunking..
    %Ironically, Lashley's student, the founder of CNS propoes PSs..

\section{Hebbian theory}

  In the 1940s, Donald Hebb, a Canadian psychologist and neurophysiologist
  stated that the problem of understanding behavior is the problem of
  understanding the total action of the nervous system and vice versa. He was
  the first to apply the principles of the neuronal doctrine \cite{Cajal1894}
  in a coherent framework in an attempt to explain the mechanisms behind the
  thought processes. In his seminal work, \cite{Hebb49} introduced three
  concepts that are widely used nowadays in the neuroscience community: Hebbian
  learning, cell assembly, and phase sequence. Hebb suggested that neurons that
  are repeatedly and synchronously activated create new synapses between
  themselves, or increase the synaptic efficacies if connections already exist.
  Thus, neurons that fire together form strong connections and organize into
  assemblies representing abstract mental concepts, now known as `cell
  assemblies' or `Hebbian assemblies'. Once an assembly is activated, it can
  ignite associated concepts by activating the corresponding assemblies, thus
  forming a sequence of activation or a `phase sequence'.
   
  figure from Hebb's book here!

  These concepts led to the development of the connectionism as movement in
  neuroscience, cognitive sciences and philosophy. Hebb's theory had a huge
  influence on the early-day machine learning development, and in particular in
  the research in artificial neural networks. For example, the Hopfield
  network, a neural network model consisting of binary neurons as a content
  addressable memory model \citep{Hopfield1982} uses Hebbian learning as a rule
  to adjust the connection weights. Another technique based on the Hebbian
  theory is the Oja's rule for unsupervised learning, which can extract the
  main features (principal components) of datasets \citep{Oja1982}.

  In the following paragraphs I make a brief review on the literature that
  followed Hebb's theory. In particular, I focus on neurophysiological studies
  that shed some light on the learning processes and the representation and
  detection of cell assemblies in the brain.
  
  \subsection{Hebbian learning: fire together, wire together} \footnote{A
    paraphrase from "neurons wire together if they fire together" \citep{Lowel1992}}

    The idea that the formation of new memories does not require new neurons is
    old. \cite{Cajal1894} suggested that for the creation or change of memories
    the brain might simply strengthen old synapses instead. This hypothesis stood
    the test of time and now is known as the synaptic plasticity and memory
    hypothesis \citep{Martin2000, Takeuchi2014}. Half a century later Hebb
    postulated that if two neurons are connected and fire in close temporal
    proximity repetitively, the efficacy of the synapses connecting them
    increases, or that even new synaptic connections could be formed, a learning
    rule known as `Hebbian learning'.

    The first demonstration of the plastic nature of synapses was demonstrated on
    anesthetized rabbits by \cite{Lomo1966} who found that a high-frequency
    stimulation to the presynaptic fibers in the perforant pathway increases the
    postsynaptic potentials (PSPs) measured in the dentate gyrus. Moreover, these
    changes in the synaptic efficacies lasted for long time periods, in the order of
    hours. The idea that the synaptic facilitation might depend on the precise
    timing of activation was proposed by \cite{Taylor1973} who stated that a
    presynaptic spike shortly before the postsynaptic activity would facilitate
    the synaptic efficacy. In a computational model, \cite{Gerstner1996} showed
    that a sub-millisecond plasticity rule depending on the exact timing of pre-
    and postsynaptic firing can indeed lead to a Hebbian learning. In the
    following year in experimental work, \cite{Markram1997} demonstrated that a
    synapse can be differentially up- or down-regulated depending on the precise
    time difference of the synaptic activation and the postsynaptic action
    potential. The spike-timing dependent plasticity (STDP) was described in a
    greater detail by \cite{Bi1998} who showed an asymmetric temporal window of
    plasticity between pyramidal neurons in hippocampal culture (see
    Figure~\ref{fig:stdp-exp}). According to this rule, the order of spiking
    determines the sign and magnitude of synaptic change; a presynaptic spike
    followed closely by a postsynaptic leads to long-term synaptic potentiation,
    while a reverse firing order leads to long-term depression. The exact
    biological mechanism behind STDP remain elusive.

    figure from Bi and Poo and more
    %\label{fig:stdp-exp}

    The classical asymmetric exponential temporal window has been replicated by
    a number of studies \citep[e.g.,][]{Bi1998, Debanne1998, Zhang1998}.
    However, because of the used techniques, the results are met with some
    reservations. For example, \cite{Lisman2005} points out that in the
    aforementioned studies, the postsynaptic firing was induced by a current
    injection instead of more natural synaptic inputs. Spikes evoked by the
    postsynaptic potentials alone at low frequency does not lead to any
    synaptic potentiation \citep{Wittenberg2006} suggesting that additional
    factors might be involved in the plasticity processes. Further work has
    shown that the conventional STDP window \citep{Bi1998, Kempter1999} is not
    universal for all synapses. Various STDP temporal windows have been shown
    depending on the brain region, preparation, stimulation protocol, and the
    type of pre/post-synaptic neurons \citep{Feldman2012, Vogels2013}. Recently
    \cite{Mishra2016} demonstrated that in slices of mature rats, the
    potentiation of CA3-CA3 recurrent excitatory synapses is independent on the
    temporal order of stimulation resulting in a symmetric STDP curve. The
    authors argue that the symmetric STDP curve (see Figure~\ref{fig:stdp-exp})
    allows for a reliable storage in the associative CA3 network. In-vivo
    experiments point out that the sign of synaptic changes might also depend
    on the exact phase of stimulation during the hippocampal theta oscillation
    \citep{Hoelscher1997}.

    small final paragraph?

  \subsection{Hebbian assemblies.}
    The idea of neural assembly as basis for memory representation was proposed
    by \cite{Semon1904} who introduced the term `engram'. According to Simon,
    engrams are representations of specific stimuli, and engram complexes are the
    basis of memory traces. Later, \cite{Hebb49} defined the assembly concept
    more specifically by postulating that neurons receiving similar inputs, form
    strong connections among themselves, and organize into assemblies or engrams.
    Neurons in these assemblies are activated synchronously when the associate
    mental concept is revoked. While not specifically defined, the cell 
    assembly has been a well-accepted conceptual tool that is widely used 
    in theory and experiments.
   
    The search of the assembly representation in the brain has proved to be a
    challenging task. Many questions are still open to debate, such as what
    what is the size of an assembly, how reliable is the participation of
    neurons from one activation to another, how stable are the assemblies
    during reactivation, do assemblies overlap? Last or not least, is the
    assembly concepts too general and thus difficult to prove and disprove
    \citep{Wallace2010}? The most promising hints for the existence of
    assemblies come from the hippocampus and the early processing sensory brain
    areas where experimentators have better control over the inputs. There are
    few lines of evidence that keep this concept alive.
    %: some cells fire
    %specifically to certain sensory inputs or represent some more complex
    %mental concepts (e.g., grandmother cells \citep{}, place cells); clusters of
    %connectivity have been reported in various brain regions where cells that
    %fire together share more common neighbours than expected by chance,
    %neuronal activity can transiently and abruptly switch from one activity
    %mode to another as sign of discrete population activity \cite{Hopfield}.
    %as memories change and update..., and so do dendrites \cite{Yang2009}

    Despite the fact that random neural networks are popular model used by
    theoreticians, their appearance in biological neural networks is scarce.
    Clustered connectivity patterns have been reported in various cortical
    areas \citep{Song2005, Ko2011, Perin2011, Shimono2015} as well as in the
    hippocampus \citep{Takahashi2010, Guzman2016}. Moreover, neurons form
    bidirectional connections more often than expected from chance
    \citep{Markram1997, Song2005, Takahashi2010, Ko2011, Perin2011}. The
    distribution of synaptic weights is non-uniform. Local neocortical networks
    exhibit distributions of synaptic weights that are heavily skewed, and
    bidirectional connections are stronger than uni-directional
    \citep{Markram1997, Song2005, Buzsaki2014}. \cite{Shimono2015} has shown that
    in neuron culture, the microconnectome has different levels of clustering,
    from a few neurons up to hundreds of neurons. The authors suggested that
    the different levels of organisation leads to different levels of
    robustness.

    How does connectivity reflect on the activity? It has been shown that
    ongoing spontaneous activity in the cat visual cortex switches between
    different states some of which corresponds to the orientation maps of
    neurons \citep{Kenet2003}. This results suggest that that neurons
    preference tuning is not purely due to the sensory input but also reflects
    some intrinsic network state. Comparison between spontaneous and evoked
    activity in the auditory and somatosensory cortices of rats showed that
    network dynamics is largely conserved between states, and that activity was
    drawn from a rather limited `vocabulary' \citep{Luczak2009, Luczak2012}.
    
    Although direct link between connectivity and activity is rather sparse,
    there are some evidences that neurons in functional assemblies (groups of
    neurons that are co-activated simultaneously) are highly connected.
    \cite{Takahashi2010} has shown that in organotopic slices neurons that
    exhibit highly correlated activity are connected with higher probability
    than uncorrelated neurons. Moreover connected neurons share common input
    and output neurons than expected by chance. In-vivo work by \cite{Ko2011}
    in L2/3 of the mouse primary visual cortex revealed that cells with similar
    preference have bigger probability of uni-and-bi directional connections.
    Moreover, \cite{Cossell2015} showed that the synaptic strength between
    these neurons varies over 2 orders of magnitude, and the strongest
    connections are between neurons with very correlated responses, while weak
    synapses are between neurons with uncorrelated responses. Although hugely
    outnumbered, the strong inputs disproportionally controlled the response of
    neurons. 

    Another feature of Hebbian assemblies predicted by the theory is the
    discrete activation of neural populations as the network activity gets into
    various attractor states \cite{Hopfield1982}. Attractors defined as
    all-or-none states in the network activity, where states close to these
    points are attracted to them. Such attractor-like activity that posses
    reliably revoked spatio-temporal patterns have been observed in slices
    \cite[e.g.,][]{Cossart2003, MacLean2005}. More recently,
    \cite{Bathellier2012} have shown that sound stimuli are evoking
    attractor-like dynamics with abrupt switching between the different
    discrete modes in superficial layers of auditory cortex. The discrete modes
    constitute of partially overlapping subpopulations where same neurons can
    take part in a few assemblies, and assemblies interact in competitive
    fashion. Due to the complexity of stimuli representation, such measures are
    more challenging in the higher cortex. By projecting the measured neuronal
    activity in high-dimensional state space by kernel methods (PCA),
    \cite{Balaguer2011} have shown that on-going activity in the higher
    cortices posses attractor-like dynamics. Abrupt transitions between
    attractor states have been also reported in experiments where the external
    sensory cues were gradually changed. For example, continuously varying an
    odor results in abrupt changes in the odor representation in the olfactory
    bulb of zebrafish \citep{Niessing2010}. Or changing gradually the
    environment evokes place cells abruptly and simultaneously to change
    representation \citep{Wills2005}. As shown in virtual teleportation
    experiments, one cycle of the theta oscillation is a temporal unit for
    expressing an attractor state in the hippocampus \citep{Jezek2011}. It is
    not known yet whether this discretization of activity is due to purely
    internal dynamics or the inputs to the hippocampus are already discrete
    with theta-cycle resolution. 

    %Okun2012: neuronal ensembles change as the network state changes
    %Rothschield: A1, functional organazation of neurons into subnetowrks with some discrete activty

    A lot of work has been devoted into capturing this `holly grail' in
    neuroscience. Numerous evidence point out that fine-scaled subnetworks
    specialise in processing similar information where correlation in activity
    is correlated with connectivity. While the assemblies in the primary
    cortical areas (V1, A1, etc.) are constituted by neurons that are spatially
    close to each other \cite[e.g.,][]{Bathellier2012, Cossell2015}, in the
    hippocampus the assemblies are spatially distributed \citep{Guzman2016}.
    %Both the hippocampus and the primary cortical areas seems to be populated with competitive assemblies \cite{?}.
    However, we do not have a clear
    picture of a more general syntax explaining how these subnetworks can
    interact, especially in higher-brain areas. \cite{Hebb49} proposed that
    assemblies organize in phase (or assembly) sequences, and these sequences
    themselves constitute sequence of activation. Relaxing the definition of a
    cell assembly, \cite{Buzsaki2010} suggests that hierarchical organization
    of cell assemblies may constitute syntactical rules that define first-order
    and higher-order relationship. Cell assemblies are then defined not by
    connectivity but by their synchronous activation during a concept
    representation. \cite{Pulvermuller2010} speculates that for the information
    in the brain, the assembly syntax might be tightly related to the
    linguistic syntax that we use. 


  %neural syntax facilitates the interaction between different hierarchies of assembly sequences..

  %Harris et al: theory of harries neocortex probing for actiavations?
 
  %possible pluses: reliability, more reliable activity trasmission than single neurons/readers as tolerates variability in single neuron firing; possibly more adaptable in learning new concepts and keep the old ones

  %Jensen and Lisman96, Buzsaki2010 neural words might be short assembly sequences that fire in single gamma cycles..
..
  %assemblies, physiology, fmri?
  
    %%%%%%kkk

    %perin2011: lego-blocks fro assemblies...
    %with similar receptive fields form assemblies that 
    %can be activated either spontanouesly or by weak input stimulus 

    %Miyashita88b: neurons fire specifically to objects in neocortex of 
    %there should be older studies on that!!

    %???
    %Liu2012: fear memory actiavtion through optogenetics
    %also others that did that???

    % maybe that should go at the end of this subsubsubsusbsubsusbsusbsusbsusbsubsubsubusbsubsubsection
  %A lot of work has been devoted into capturing this `holly grail' in
  %neuroscience. While numerous evidence point out that neurons might indeed
  %organize into functional assemblies, even more questions about the operation
  %of the nervous function arise.

  %miyashita88a: fractal recognition neurons in monkeys
  %we become what we have learned to recognize, 

  %experiment to try with more common common and uncommon input and see how neurons respond
  %prediction: high firing for common and low for uncommon (unnatural)
  %fractals are probably some of the most natural?
  
  %some stability stuff?
  %fernando-ruiz: functional assemblies in CA3 in-vivo; changing participants..

  %Question is why does the old state still manifistate itself after the mouse is the new environment, is this the left-over of some working memory, what is the basis of it. Is it from hippocampus or from external working mewmory input? Jezek2011 teleportation


%A big question is how stable are the representations of stuff in brain, and if they change how they change 
%as a result of experience; consolidation of memories/associations, how does it change the representation of these objects, in hippocampus or in other regions of neocortex???


  %refs
  %H Markram,  Physiology and anatomy of synaptic connections between thick tufted pyramidal neurones in the developing rat neocortex.
 
  %Semon, R. (1904). Die Mneme [English Translation: The Memory]. Leipzig:
%Wilhelm Engelmann.

  \subsection{Phase sequences.}
    The third less known principle proposed by \cite{Hebb49} is about `phase
    sequences': once an assembly is activated, i.e., the underling neurons are
    firing, its activation would propagate and activate another assembly, thus
    mental concepts would ignite associated concepts. Hebb suggested that such
    sequential activation of assemblies of neurons underlies our most complex
    mental processes. Without going into much details, he suggested that phase
    sequences can interact with each other and organize more complex
    hierarchies and sequences. 

    Various brain areas reveal spatiotemporal activity patterns that repeat
    over time \cite[e.g.,][]{Wilson1994, Kenet2003, Berkes2011}. Therefore, it
    is common to find groups of neurons that fire reliably in a temporal order
    during the execution of the temporal patterns. Moreover, neurons do not
    always fire in a single burst but show more various patterns such as ramped
    activity, or more complex or chaotic firing. The microcircuits
    underlying such activity are largely unknown, and there are scarce
    evidences pointing to a sequential activation of neural assemblies. An
    interesting idea by \cite{Goldman2009} states that decomposition the
    connectivity matrix of a recurrent network in Schur modes allows to project
    the activity as a feedforward interaction between the activity modes. Thus,
    with the appropriate projection, any recurrent network can be viewed as a
    `feedforward network in disguise'.
    %Howevwe FF nets are powerful model as their relativily simplicirty compared to RNN
    %and are largely applied in artificial nets : MLP, DL, ?. while RNns are sparsely applied
    %Hopfield, LSM, ESN. FF are easy to grasp while RNN are too complex.. Goldman here?

    Some of the most striking examples of precise sequences of neural activity
    come from songbirds. Male songbirds such as the zebra finch perform complex
    stereotypical songs consisting of variable patterns on multiple timescales.
    The songs are learned from a tutor bird and are extensively rehearsed for
    around 60 days until deployed in practice upon reaching sexual maturity
    \citep{George1995, Doupe2004}. It is believed that song syllables and tempo
    motifs are stored in a forebrain nucleus called HVC (the abbreviation does
    not stand for anything; in the past it was Hyperstriatum Ventrale pars
    Caudalis, or High Vocal Area) where neurons are firing selectively to
    sounds, syllables, or sequence of syllables \citep{Yu1996}. The HVC premotor
    neurons are shown to fire sparsely and very precisely in stereotypical
    sequences of bursts with very small jitter ($<1\,\rm ms$) when aligned with
    the performed song \citep{Hahnloser2002}. The same sequences of activity
    are replayed offline during sleep which is believed to support the memory
    consolidation of songs \citep{Dave2000}. While most models focus on
    feedforward networks for learning and performing of songs,
    \cite{Kosche2015} have shown that the song production relies on patterned
    inhibition and excitation for the generation of the song sequence. The
    circuit generating these sequences is still illusive.  While the
    connectivity within the recurrent connectivity is mostly unknown, the HVC
    receives feedback feedback from upstream areas during song generation
    \citep{Hamaguchi2012}.

    picture from hahnloser

    For the assembly sequence hypothesis, it is important that the activity of
    one assembly synchronously drives the activity in another assembly. Such
    coding is compatible with the temporal code where the exact spike timing in
    relation to other neurons is crucial for the information transmission.
    Indeed, sequences of neural activity on fast, millisecond time scales have
    been reported in various neuronal circuits such as, just to name a few,
    insects \citep{Fushiki2016}, fishes \citep{Romano2015}, in mammalian cortex
    \citep{Euston2007} and hippocampus \citep{Lee2002}, as well as in-vitro
    preparations \citep{Segev2004, Kruskal2013}. Sequences on a slow-time
    scales, in orders of seconds has been reported in the hippocampus during
    REM sleep \citep{Louie2001}. The mechanisms behind such replays are
    unknown.

    Sequence replays depend on the behavior state \cite{Almenida2014}, and
    reverberations are enhanced mostly in the desynchronized states
    \cite{Contreras2013, Buzsaki1983} when the local circuit activity is
    presumably intrinsically driven. xxxx 
    
\section{Hippocampus: a brief survey}

  Time is so intimate for us that it is difficult to imagine to live without
  it. We live in a dynamical system that constantly changes the environment
  that we perceive and so we change as well. As these changes reflect past
  events, they can be generalized as memories that can be in various levels,
  such as physical, genetic, neurophysiological, etc. When we use the word
  `memory' in everyday talk, usually we refer particularly to the declarative
  memory that includes episodic (spatial and autobiographical experiences and
  events) and semantic (vocabulary, facts, concepts, etc.) types of memories.
  In humans and other mammals, the hippocampus is a vital brain structure for
  storing the past in the form of declarative memories and planning for the
  future.
  %Hp is a remarkable structure that supports the formation of new declarative memories.

  The aim of this section is to give the reader a basic picture of the
  hippocampus. I start with a brief description of the hippocampus anatomy and
  then dive into its functional role.

  \subsection{Anatomy}
    The hippocampus is first described by Julius Caesar Aranzi in 1564 and is
    named due to its resemblance to the seahorse (from Greek $\iota \pi \pi o
    \kappa \alpha \mu \pi o \varsigma$: seahorse, see Figure~\ref{fig:hp}). The
    hippocampus is a subcortical structure, part of the limbic system. The
    hippocampal formation is remarkably preserved across mammalian species
    \citep{Manns2006, Clark2013}. Mammals have 2 hippocampi located bilaterally
    in the medial temporal lobes wrapped by the cerebral cortex (see
    Figure~\ref{fig:hp}). The hippocampus  proper is one of the most
    extensively studied region in the mammalian brain in terms of anatomy,
    electrophysiology, and behavior function.

    pic of hp compared to hp; hp in brain; hp loop; hp layers?
    \label{fig:hp}

    The hippocampal formation consists of the hippocampus itself, the
    subiculum, the presubiculum, the parasubiculum, and the entorhinal cortex
    which provides the major input and is the main output of the hippocampus.
    The hippocampus can be divided in dentate gyrus, and the cornu ammonis
    regions (namely, CA3, CA2, and CA1). The hippocampus has gained popularity
    in neurophysiology partly because of the unidirectional transmission of
    information from dentate gyrus to the CA regions which facilitates the
    study of the local microcircuits. The main input to the hippocampus is
    provided by the entorhinal cortex (EC). Activity from the superficial
    layers of the EC makes a round in the entorhinal-hippocampal (or
    trisynaptic) loop and projects back to the deep layers of the EC
    (Figure~\ref{fig:hp}, specify where!!). Much like the cortex, the
    hippocampal subregions are organized in well-defined laminar structure
    where different layers are populated by different cell types, and the
    different synaptic inputs are confined to specific layers. 

    \textbf{Dentate gyrus microcircuit}\,\,
    The dentate gyrus (DG) consists of three neuronal layers: molecular,
    granular, and polymorphic. The principal excitatory cells, the granule
    cells are located in the granular layer. The main input to the DG comes
    through the perforant path from layer II of the entorhinal cortex
    \citep{Squire1992} while the output consist of unidirectional connections
    to the CA3 region through the mossy fiber (MF) synapses. Due to their
    extremely high efficacies where a single presynaptic spike can drive the
    postsynaptic neurons to fire, the MF synapses are also known as
    `detonators' as

    Another peculiar feature of the DG is the fact that granule cells are
    generated through the whole life in rodents as well as in humans. Due to
    the neurogenesis, the number of neurons in DG can vary, and subjects that
    use more extensively spatial navigation tend to have larger number of
    granule cells. Moreover, the size of the DG correlates with spatial memory
    \citep{Maguire2000, Spalding2013}.

    \textbf{CA3 microcircuit}\,\,
    The CA3 hippocampal area is an important focus of this thesis. Like the
    rest of the hippocampal subregions, CA3 posses a well-defined laminar
    structure that confines the localization of the different neuron types and
    inputs (see Figure~\ref{fig:hp}). The deepest layer of the CA regions is
    the lacunosum moleculare (sl-m), followed by stratum radiatum (sr),
    pyramidal cell layer (pcl). The most superficial layer is the stratum
    oriens (so). All layers are populated by inhibitory interneurons. There are
    over 20 types of interneurons that can be classified depending on the
    location, morphology and genetic expression \citep{Maccaferri2003,
    Klausberger2008}. The principal cells, the pyramidal neurons are located in
    the pyramidal layer and around $90\%$ of all the cells in the area. Their
    number in rats is $\sim 3 \cdot 10^5$ \citep{Boss1985, Boss1987} in CA3,
    while the human hippocampus has around $2.7 \cdot 10^6$ neurons in CA3 and.
    Pyramidal neurons extend their apical dendritic trees in the superficial
    layers, while the basal branches are in the deep cell layers. The axons of
    pyramidal neurons can innervate around 70\% of the dorsal-ventral axis of
    the hippocampus and project to other pyramids and interneurons in the
    stratum radiatum and oriens \citep{Sik1993, Li1994}.

    A major input to the CA3 comes through the MF synapses from the dentate
    gyrus. The MF terminals form a distinguishable layer within the stratum
    radiatum, sometimes referred to as stratum lucidum. Other major input comes
    from the entorhinal cortex, layer II. Interestingly, the axons reaching the
    CA3/2 regions project also to the DG which implies that similar information
    from EC, layer II is conveyed to the CA3 region from 2 different sources.

    The CA3 hippocampal area was considered to be densely connected 
    \cite[$\sim3 \% $ in][]{Miles1986} relatively to the CA1, but more recent evidences show that %$]{hui}
    the connectivity between principal cells is around 1\% \citep{Guzman2016}.
    The major output of the CA3 are the excitatory axonal projections to the
    ipsilateral and contralateral CA1 area via the Schaffer collaterals
    \citep{Finnerty1993}. The reader is kindly referred to \cite{Duigou2014} for a
    more detailed description and analysis of the connectivity within the
    CA3 region.
    %plastic conenctivity who????

    \textbf{CA1 microcircuit}\,\,
    The CA1 hippocampal region has a similar laminar structure as the CA3.
    However, the DG input is absent and the entorhinal input originates from a deeper
    layer, EC, layer III. The principal neurons are more densely packed than in
    CA3, and there are around $3.5\cdot 10^5$ neurons \citep{West1991} and $20.9
    \cdot 10^6$ in humans \citep{Simic1997}. For a more in-depth analysis of
    the recent findings on anatomy and connectivity of different cell types in
    the CA1 region, I kindly refer the reader to \cite{Bezaire2013}.

    \textbf{Entorhinal cortex}\,\,
    The entorhinal cortex provides the major interface to the hippocampus. The
    EC is part of the neocortex with the typical 6-layered structure. Principal
    neurons are pyramids and stellate cells and populate layers II, III, and V.
    As mentioned above principal cells from layers II of EC are projecting to
    the dentate gyrus and CA3, while layer III neurons are innervating the CA1
    region and the subiculum. Inputs from the hippocampus are impinging on the
    deeper layer V of the EC.
    %mcNaughton2006????

    %1 DG GC to CA3: 14 synapses; 1 CA3 PC gets 46 inputs from DG GC to CA3 Amaral1990
    %12000 recurrent CA3 synapses Amaral1990
    %3750 PP inputs Amaral90

  \subsection{The role of the hippocampus in the human brain}
    In the past, the hippocampus was assumed to be involved in various tasks. A
    dominant hypothesis until the end of 1930s suggested that its involvement
    in olfactory processing \citep{Andersen2007}. Later hypothesis assigned
    roles in emotions and attention. Some earlier results from Brown and
    Sch\"{a}fer in 1888 and Bekhterev in 1900 hinted to the involvement of the
    hippocampus in memory \citep{Andersen2007}, but it was not a popular view
    until \cite{Scoville1957} presented a study on the amnesic patient H.M.

    Henry Molaison, known as patient H.M. during most of his life, is probably
    the most influential patient in neuroscience. In an attempt to be cured
    from epileptic seizures, he underwent a brain surgery in 1953 in which most
    of his hippocampi, amygdalae and the surrounding entorhinal and
    parahippocampal cortices were removed. The operation was successful in its
    initial goal that he no longer suffered from seizures, however, at the
    price of some `side effects'. After the surgery H.M. was diagnosed with
    anterograde amnesia, i.e., the inability to form new (declarative)
    memories. Moreover, he suffered from temporally graded retrograde amnesia,
    i.e., he could not or had difficulties to recall events that took place
    shortly before the surgery while more distant memories from the past were
    intact. While subjected to numerous psychological experiments until his
    death in 2008, H.M. showed intact intellectual abilities and short-term
    memory. He revealed that the memory system is not a single entity but there
    are multiple subsystems with possibly different neural origins. For
    instance, he could acquire new implicit memories when subjected to
    repetition priming \citep{Corkin2002}. Moreover, he showed intact motor
    learning. So in a thought experiment, he could learn to play ukulele, for
    example, while not possessing any knowledge and memory that he is able to
    play. Studied for more than 50 years, H.M. pointed out the importance of the 
    hippocampus for the formation of new episodic and semantic memories.

    Work with other hippocampal amnesic patients has revealed some
    dissociation between the two types of declarative memories, i.e., episodic
    (or autobiographical) and semantic. While the overall recall of distant
    memories is intact in amnesic patients, the recall of the autobiographical
    memories are somewhat not as vivid and detailed as in control subjects.
    \cite{Moscovitch2005} suggested that the hippocampus is actively engaged
    during the recollection of episodic memories, and possibly during the
    recall of semantic memories that have some autobiographical elements.

    The disfunction of the hippocampal formation causes not only impairment of
    remembering the past but also the inability to see in the future. Work with
    amnesic patient with bilateral hippocampal damage has suggested the
    involvement of the hippocampus in the process of planning and imagining new
    experiences \citep{Klein2002, Hassabis2007}. Overall, these results give
    the hippocampus a very peculiar function as a `time machine' in our
    cognition. Apart from simply storing experiences and semantic knowledge,
    the hippocampus also lets us mentally explore past or future experiences.
    Such view naturally evokes the question whether H.M. or other amnesic
    patients ever experience mind-wandering.

    %one-shot learning: nakazawa2003; feng2015
    %involvement in spatial orientation..hp lesion leads to spatial disorientation (few interesting refs from jorge?)
    
  \subsection{Neural correlates to behavior} 
    After H.M. sparked the interest to the human hippocampus, a numerous
    number of electrophysiological studies has been conducted in animals in the
    search of the neuronal basis of memory.

    \cite{OKeefe1971} for the first time showed that some hippocampal pyramidal
    neurons in behaving rats are activated while the animal is at a specific
    location. As such cells code for the position in space, later they were
    named `place cells'. Following experiments have shown that a subset of
    hippocampal and entorhinal neurons can code for head direction
    \citep{Taube1990a, Taube1990b}, odors \citep{Wood1999}, environment
    boundaries \citep{Barry2006}, particular objects \citep{Manns2009}. The
    discovery of cells in the entorhinal cortex that fire at multiple locations
    in an environment and forming a particular hexagonal grid
    \citep{Hafting2005} have attracted the attention in the field of
    neuroscience. In a very recent reports, \cite{Sarel2017} described
    hippocampal cells in bats that code for the vectorial representation of
    goal location. Such firing pattern have been predicted previously by theory
    \citep{Stemmler2015}. So far, the rodent hippocampus has shown remarkable
    repertoire of neural codes for spatial navigation that does not simply
    reflect the sensory input but reveals a more abstract representation of the
    environment.
  
    There are some evidences for the existence of place cells and grid cells
    also in humans while exploring virtual environments \citep{Ekstrom2003,
    Jacobs2013}. Experiments with monkeys have revealed the existence
    view-responsive neurons, that is, place-like and grid-like activity of
    neurons is evoked when the animal is looking at a particular location in
    the visual field \citep{Rolls1995, Killian2012}. This data suggests that
    hippocampal activity during navigation does not code only the current
    position of the subject but can also encode virtual exploration of
    environment.
    
    Recordings from the hippocampal formation of epileptic patients have shown
    that some neurons fire very selectively to particular objects, landmarks or
    individuals \citep{Quiroga2005}. The firing of these cells is strikingly
    invariant as, for example, they are activated when various photos of a
    Jeniffer Aniston are presented, irrespectively of the orientation and the
    angle of shooting. Moreover, the corresponding cells are even activated
    when only the written name is presented to the subject. Such cells are
    called `grandmother neurons' and are unofficially known as `Jennifer
    Aniston neurons'.

    While the aforementioned studies show strong correlations between
    hippocampal activity and behavior, what is the role of such representations
    for the cognition, and for the memory in particular? With the advance of
    the neuroimaging techniques, and the optogenetics specifically, now
    researchers are able to track the activity of large population of neurons
    (e.g., thousands of cells) and manipulate their behavior during
    experiments. Tagging active cells during an association learning task, one
    can isolate the neural assembly that is activated during the learning. A
    following activation or inhibition of the assembly can evoke or suppress
    the learned response \citep{Cowansage2014, Tanaka2014}. These results
    provide an evidence for the active role of the hippocampus in the neural
    representations during behaviour.

  \subsection{Oscillations}
    % make a nice intro; check buzsaki's rhythms of braino
    Brain rhythms are omnipresent in the brain. Oscillations with various
    frequency can be detected virtually in any brain region using extracellular
    electrodes or electroencephalography (EEG). These oscillations reflect the
    underlying neuronal activity that is synchronized across large populations
    of cells and are often strongly correlated to specific behaviors that the
    subject is engaged in.

    % oscillations in hp
    The hippocampus, particularly, shows some of the most striking rhythms in
    the mammalian brain. Electrophysiological recordings \textit{in vivo}
    reveal a rich repertoire of rhythms in the local field potentials that are
    state dependent. For example, the theta oscillation ($\sim 5-10 \rm\, Hz$)
    that is amongst the most regular rhythms is associated with active
    behavior such as walking and running, and also with the rapid-eye-movement
    (REM) phase of sleep. Moreover the theta phase cycle modulates the
    amplitude of the faster gamma oscillation ($\sim 30-100 \rm\, Hz$) that
    rides on top of it. Neurons' firing is largely modulated by the oscillation
    phase. While pyramids are mostly firing at the trough of the theta cycle,
    the various interneurons have different preferred phases
    \citep{Klausberger2008, Klausberger2009}. Moreover, neurons are differentially modulated by
    the gamma oscillation.

    pic of theta, theta-gamma, pp

    As mentioned above, during exploration pyramidal neurons are modulated by
    the current position of the animal, i.e., place cells. Interestingly, a
    place cell's firing pattern shows a peculiar modulations in respect to the
    local-field theta oscillation. When the animal is entering the
    corresponding place field, the neuron starts to fire at a particular phase
    of the theta cycle, the so called entering phase. As the animal progresses
    within the field, the neuron fires at earlier phase in each subsequent
    theta cycle. The range of phases that a neuron is precessing does not
    exceed the length of a theta cycle. This advance of firing phase in respect
    to the theta oscillation is known as `phase precession' \citep{OKeefe1993}.
    Thus, in addition to the information coded in the firing rates of neurons
    (i.e., the rate code), there is an additional information about the exact
    location of the animal carried in the exact spike timing in respect to the
    theta oscillation (i.e., the temporal code).
    
    Phase precession has been in the focus of research due to its assumed role
    in learning, and particularly, it has been suspected to be a mechanism for
    bridging the temporal gap between the behavior time scales (in the order of
    seconds) to the physiological time scales at which synaptic plasticity
    operates (order of milliseconds to tens of milliseconds) \citep{Bi1998}. For
    example, disruption of phase precession through a systemic administration
    of cannabinoid receptor agonist leaves the spatial selectivity of
    hippocampal neurons intact, but however, impairs the performance of rats on
    memory tasks \citep{Robbe2009}.
    
    When the animal traverses partially overlapping place fields, the
    corresponding place cells are sequentially activated. Moreover, as place
    cells phase precess, the sequential spike order is preserved in the firing phase
    difference on a fine time scale (Figure~\ref{fig:theta,xxx}). Moreover, the
    sequential spiking of neurons is more reliable than the spiking phase of
    single neurons \citep{Dragoi2006}.  Such firing sequence within a theta
    cycle are called `theta sequences'. The temporal compression of
    sequences with in a theta cycle is assumed to be a crucial mechanism for
    the formation of memories \cite{??}, and in particular of the single-trial
    learning \cite{Rutishauser2006}. 
    
    What is the relation between phase precession and the theta sequence, and
    are they the two sides of the same phenomenon? \cite{Feng2015} observed
    phase precession in neurons during the first exposure of animals in novel
    environment, but however, reported the absence of theta sequences. From the
    second trial on, stable and reliable theta sequences were present. These
    result suggest that a single-trial experience is needed to form the
    sequence in the memory. Theta sequences, however, do not cover uniformly
    the whole environment, but are segmented in their representations. \cite{Gupta2012} 
    showed that different theta sequences reliably represent chunks of the environment,
    frequently bounded by the space between landmarks.
    % looking in back in the past, or forward in the future

    Sequence replay is observed also during other prominent hippocampal
    oscillations, the sharp-wave ripples. Due to the central place of this
    phenomenon in this thesis, the following Section~\ref{sec:swr} is dedicated
    to their more detailed description.

  \subsection{Theories on the role of the hippocampus} 

    start with cognitive map?

    \citep{Marr1971} proposed a general theory for the function of the
    hippocampus that is still in the base of the majority of current
    hippocampal models. according to his model the hippocampus acts as a
    temporal storage for memories of events that are rapidly stored after a
    single-shot learning. later, a fraction of these memories are imprinted in
    the neocortex for long-term storage through a process called `memory
    consolidation'. moreover, he assigned different roles to the hippocampal
    subregions. for example, the dentate gyrus performs pattern separation,
    i.e., similar input patterns are matched to dissimilar output. on the other
    hand, the ca3 region performs the function of pattern completion, i.e.,
    recovery of a stored pattern given only a fragment of the input. while
    marr's theory is a rather theoretical study, \citep{buzsaki89} presented
    the `two-stage memory model' that matches better the electrophysiological
    phenomena found in the hippocampus. further theoretical models on memory
    consolidation with various degree of details and focus on the type of
    memories have been presented \cite{e.g.,}{}{squire92, rolls96,
    eichenbaum2004, jensen2005, hopfield2010, cheng2013b}
    tolman48: cognitive map

    fig from rolls96 (theory on hp f in memory)

    most (if not all) theories describing the function of the hippocampus on
    memory are founded on the synaptic plasticity and memory hypothesis
    \citep{Martin2000, Takeuchi2014}. an underlying theme in the majority of
    hippocampal models is the fact that the hippocampal formation is located on
    the top of a pyramidal hierarchy of information flow. this position
    provides the hippocampus with access to various cortical areas and with the
    opportunity to link neocortical representations from different modalities
    in single episodes. 

    to check:
    rolls and kesner 2006; hasselmo 2012; buzsaki2010

    In the brain however, there is a rich repertoire of rhythms and oscillations that synchronise
    distal brain areas.  theta-gamma nesting, time window of 20-30 ms, compatible with the epsp width.
    suggested that during a gamma cycle the activation of n unitary semantic block the assembly, and during theta
    a sentence consisting of ~7 words..

    %rutishauser2006: single-shot learning in hp

    %most theories rely on various neocortical representations that get
    %activates by hp simultaneously and thus get more connected

    %animals with hippocampal lesions perform on associations tasks where US and CS are 
    %simultaneously shown but fail when an interval separates both (refs in manns2005)
    %recognition memory also works for short periods but fails for longer periods..

\section{Sharp-wave ripples}

  Sharp-wave ripples are remarkable hippocampal population bursts during which
  a large number of neurons are orchestrated in precise firing. The sharp waves
  are reflected in brief (50--$100 \rm\,ms$) and large ($>1 \rm\, mV$)
  amplitude deflections observed in the local-field in the stratum radiatum of
  the CA regions (see Figure~\ref{fig:swr}) with a varying incidence range
  (0.01--$3 \rm \, events/sec$). The complimentary phenomenon, the ripples are
  fast (120--$200 \rm\, Hz$) oscillations that are most prominent in the
  pyramidal cell layer in the CA1 region. Sharp-wave ripples play an important
  role in memory processes such as learning, retrval, consolidation, as well as
  planning.

  The first reports of the sharp wave phenomenon come from rabbits
  \citep{Stumpf1965} and dogs \citep{Yoshii1966}, monkeys \citep{Freemon1969}, and
  humans \cite{Freemon1970}, and the first classification have given the name
  `large irregular activity' \citep{Vanderwolf1969}. \cite{Buzsaki1992} first
  analysed and named the fast oscillation component, the ripple. Afterwards,
  sharp-wave ripples have been detected in every tested mammalian specie. 

  In what follows, I am going to describe the sharp-wave ripples and some of
  their main properties with a varying degree of detail. For the curious reader
  looking for a more in-depth information concerning SWRs, I highly recommend
  the recent \textit{opus magnum} by \citep{Buzsaki2015}.

  fig on swr

  \subsection{Behavior correlates}

    In-vivo sharp-wave ripples (SWRs) can be observed during sleep and
    wakefulness. In the awake state, SWRs occur predominantly during `off-line'
    resting behaviors such as drinking, feeding, grooming, whisking or being
    still; but can also be seen during the `exploratory' theta-dominated states
    \citep{Oneill2006}. During sleep, SWRs are observed in the non rapid eye
    movement (nREM) phase of the sleep, the slow-wave sleep (SWS). The SWR
    events are associated with a massive population bursts of activity where
    large numbers of cells are synchronously active. The fraction of engaded
    cells can vary from 1\% to 40\% but a typical number is 10\%
    \citep{Mizuseki2013}. SWRs co occur with the thalamo-cortical spindles. It
    is hypothesised that the this co-activation facilitates the communication
    between the hippocampus and the neo-cortex \citep{Sirota2003}.
    
    While the exact functions and mechanisms for the generation of SWRs are
    still illusive, it is known that they are involved in the formation and
    consolidation of new memories. According to the dominant theory in the
    field, i.e., the two-stage memory hypothesis \citep{Buzsaki1989}, SWRs perform
    information transfer of memory traces that are temporally stored in the
    hippocampus to the distributed neocortical network. This hypothesis is
    supported by a few lines of experimental observations. On the one hand,
    higher incidence of spindles and SWRs is observed after learning \citep{Eschenko2006, Eschenko2008,
    Girardeau2014}, and the number of successfully recalled items in a learning
    task is correlated to the incidence of sharp waves measured in the rhinal
    cortex \citep{Axmacher2008}. Moreover, boosting slow-wave oscillations
    through transcranial stimulation during sleep potentiates memory
    performance in humans \citep{Marshall2006}, and presenting odor cues during
    SWS, but not during REM sleep or wakefulness enhance declarative memories
    that were formed during the odor exposure \citep{Rasch2007}. On the other
    hand, disruption of SWRs during sleep or wakefulness impairs the
    performance of rats on hippocampus-dependent spatial tasks
    \citep{Girardeau2009, Jadhav2012} presumably deteriorating the memory
    consolidation process and the planning of future actions, respectively.

    Another line of observation supporting the hypothesis that the hippocampus
    acts as a temporal storage of memory traces is the replay of behavior
    sequences that takes place during sharp waves. \cite{Wilson1994} showed that
    in rats, hippocampal neurons with overlapping place fields that were
    coactive in the awake state showed strong activity correlations during
    SWRs. Later, \cite{Lee2002} demonstrated that the sequence of experienced
    events is preserved during the sharp waves but temporally compressed $\sim
    20\rm fold$. Some more recent experiments reveal that the hippocampal SWR
    replay is coordinated with a grid cell replay of sequences in the
    entorhinal cortex \citep{Olafsdottir2016}. Moreover, \cite{Oneill2017}
    demonstrated that replays in superficial layers of the EC can occur
    independently of the hippocampal sharp waves. In that last study, however,
    the EC replays followed synchronous CA1 population events suggesting that
    the replays can be nevertheless triggered from the hippocampus.
     
    An amazing amount of data supports the memory-consolidation hypothesis of
    sharp-wave ripples, but however, some experimental results have challenged
    the model. \cite{Foster2006} showed that immediately after a spatial
    experience, the most recent spatial episodes are replayed in a temporally
    reversed order. While such reverse replays are assumed to consolidate the
    most recent experiences, their appearance question the relevance the
    classical spike-timing dependent plasticity (STDP) rule with asymmetric
    window \citep{Bi98} as a mechanism for sequence storage. A somewhat bigger
    challenge to the classical model have been the reports of preplay of
    behaviour sequences. \cite{Dragoi2011} measured the neuronal activity
    during SWRs prior to animal's exposure in a novel environment and reported
    replay of place-cell sequences representing the not-yet exposed
    environment. Such results suggest that sequences are not rapidly stored in
    the CA3 region, but that they might be precoded. An estimation based on multiunit
    activity recordings predicts that the rat hippocampus might store sequences
    of about 15 different novel environments \citep{Dragoi2013}.
    
  \subsection{Generation mechanisms}
    Sharp waves ripples can be observed in the whole hippocampal region. They
    are most prominent in the CA regions revealing a stereotypical laminar
    profile signature in the local-field potential (LFP) (see
    Figure~\ref{swr:lfp}). A current-source density analysis shows that sharp
    waves are associated with a large negative deflection of the LFP (sink) in
    the stratum radiatum which reflects excitatory input impinging on the
    apical dendrites of the pyramidal neurons. The ripple component is
    prominent in the stratum pyramidale where the local potential shows a fast
    oscillation ($\sim200\rm \, Hz$) riding on the top of a positive
    deflection.

    The temporal order of activation suggest that the origin of the SW burst is
    in the CA3 region, presumably initiated by a built-up of recurrent activity
    \citep{delaPrida2006,schilglof?} and then the activity propagates to the
    CA1 region through the Schaffer collateral projections
    \citep{Csicsvari2000}. The propagation of activity from CA3 to CA1 is
    important for memory formation.  Blocking the excitatory CA3-CA1
    projections impairs the formation of hippocampus-dependent memories in
    mouse \citep{Nakashiba2008}. Interestingly, SWRs were still observable in
    the CA1 region, although with a lower incidence, ripple frequency, and
    number of ripples per event \citep{Nakashiba2009}. This results suggest
    that the CA1 region is also able to generate sharp waves independently of
    the CA3, when integrating inputs from cortical and subcortical areas.
    % of activated pyramidal neurons: from 1 to 50% 

    In-vivo data has suggested that SWRs are local hippocampal events. For
    example, decoupling the hippocampus from the rest of the brain by removing
    the main inputs led to spontaneous dynamics largely dominated by SWRs
    \citep{Buzsaki1983}, pointing out that SWRs might be a `default' network
    state for the hippocampus. In-vitro slice models have been developed
    to facilitate the study SWRs \cite[e.g.,][]{Maier2002, Maier2003,
    Kubota2003, Colgin2004}. The slice models offer clear advantages in
    investigating the synaptic, the cellular, and the network properties of
    neuronal circuit by giving more control to the experimentators. Horizontal
    slices of thickness around 300--$600\, \mu m$ are typically used for
    in-vitro studies. In-vitro SWRs show preserved main characteristics such as
    spontaneously generated events, propagation pathway, laminar profile of the
    LFP, ripple oscillations, neuron participation, and synaptic inputs. There are,
    however, also some differences: SW duration \textit{in vitro} tend to be
    shorter and to have a slightly higher ripple frequency (150-180 \,\rm Hz \textit{in
    vivo}, $\sim 200$\,\rm Hz \textit{in vitro}). It is worth mentioning that even
    \textit{in vivo} SWRs show slightly different properties depending on the
    behaviour state and even on the location of the animal \citep{Buzsaki2015}.
    %cite chapter from R and N? on more detailed comparison
    
    Regardless of the focused work of many scientific groups, the mechanisms
    behind the SWR initiation are still not well understood. As SWRs are large
    population events, traditionally it is assumed that they originate from the
    built-up of excitatory activity mainly in the CA3 region \citep{delaPrida2006,
    more}. For example, the excitation of a large number of cells in the local
    circuit by a potassium puff leads to sharp-wave events. Recent evidences,
    however, point out to the involvement of the interneuron network into the
    initial phases of the events \citep{Sasaki2014}. \cite{Schlingloff2014, Kohus2016} have shown
    that activation of parvalbumin-positive basket cells in slices can lead to
    SWR events with short delays $xxx \rm\, ms$. SWRs can occur also by inducing a
    single action potential of a CA3 pyramidal neuron \citep{Bazelot2015}. In
    this study the authors show that the induced spike was followed by putative
    interneuron firing within a small delay $<3\,\rm ms$ indicating the relevance
    of the interneurons in SW generation.

    %ca1 mini slice
    %- involvement of cells and theories on their origin
    %- number of cells, results in various sw size, subjective threshold, influencing results

    Other open questions are what does terminate the SWR event, and what
    mechanisms control the incidence of their occurrence. In the second chapter
    of this thesis I tackle these problems, however, without giving a
    definitive answer. There, in the introduction, I provide some further
    literature overview on how SWRs incidence is controlled in experimental
    settings through pharmacology and other means.


\section{Scope}
  The focus of this thesis is on the mechanisms of SWR generation. In the second chapter I
  investigate in-vitro data...in-silico model, analytical analysis of simple 2 population system
  then other stuff


