\chapter{Introduction}
  \textit{ There is a story that Simonides was dining at the house of a wealthy
    nobleman named Scopas at Crannon in Thessaly, and chanted a lyric poem
    which he had composed in honor of his host, in which he followed the custom
    of the poets by including for decorative purposes a long passage referring
    to Castor and Pollux; whereupon Scopas with excessive meanness told him he
    would pay him half the fee agreed on for the poem, and if he liked he might
    apply for the balance to his sons of Tyndaraus, as they had gone halves in
  the panegyric.  }

  \textit{ The story runs that a little later a message was brought to
    Simonides to go outside, as two young men were standing at the door who
    earnestly requested him to come out; so he rose from his seat and went out,
    and could not see anybody; but in the interval of his absence the roof of
    the hall where Scopas was giving the banquet fell in, crushing Scopas
    himself and his relations underneath the ruins and killing them; and when
    their friends wanted to bury them but were altogether unable to know  them
    apart as they had been completely crushed, the story goes that Simonides
    was enabled by his  recollection of the place in which each of them had
    been reclining at table to identify them for separate interment; and that
    this circumstance suggested to him the \textbf{discovery of the truth that
    the best aid to clearness of memory consists in orderly arrangement}.  }

  \textit{ He inferred that persons desiring \textbf{to train this faculty must
    select localities and form mental images of the facts they wish to
    remember and store those images in the localities, with the result that
    the arrangement of the localities will preserve the order of the facts,
    and the images of the facts will designate the facts themselves}, and we
    shall employ the localities and images respectively as a wax writing tablet
  and the letters written on it.}\footnote{\cite{Cicero}, Book 2, 86.352--54.}\\\\

%\section{Summary}
  This thesis deals with sharp-wave ripples and the associated sequence
  replays. The introduction is intended to give the reader some insight about
  the relevance of these topics to our everyday life. It starts with a brief
  historical overview (Section~\ref{sec:seqs}) on mental sequences and their
  place in the understanding of memory and cognition over the centuries. Then,
  Section~\ref{sec:hebb} dwells into fundamental concepts from Hebb's theory,
  such as synaptic plasticity, neural assemblies, and assembly sequences and
  their implications in theoretical neuroscience. The following
  Section~\ref{sec:hippo} is dedicated to the description the hippocampus and
  its role in memory. Then, Section~\ref{sec:swr} describes the sharp-wave
  ripple phenomenon and the replay of behaviour sequences in a greater detail.

\section{Sequences as a behavioural substrate: a historical overview}
\label{sec:seqs}
  Simonides of Ceos (556--468 BC) was both a celebrated and condemned lyric
  poet in Ancient Greek world. He is famed as the inventor of 4 letters of the
  Greek alphabet ($\eta, \, \psi, \, \xi,\,{\rm and } \,\, \omega$) and is
  considered as the first commercial poet who created songs and odes for pay.
  Moreover, he is in the focus of the oldest reference \citep{Rhetorica} for a
  mnemonic technique called `Method of Loci' or `Memory palace' (see the legend
  above from \cite{Cicero}). The method of loci has survived over the centuries
  as a technique to help orators, artists, clerks, and encyclopedists to
  remember information of virtually any quality and quantity \citep{Yates66}.
  Nowadays, the method takes a central place in the training of every memory
  champion \citep{Foer2011}.
  
  How does this mnemonic technique work? Can a naive person actually remember a
  large list of items consisting of tens or even hundreds of elements? In its
  essence, the method of loci is a mnemonic technique based on images and
  places. To deploy the method, one needs a predefined trajectory in some very
  well known spatial environment (e.g., a good starting point for beginners is
  imagining home). Then, to imprint the list, each item is sequentially placed
  in the virtual trajectory (e.g., entrance door, telephone cupboard,
  bookshelf, door to the living room, sofa, sofa table, etc.) and imagined as
  vividly as possible. It is suggested to recruit as many sensory modalities as
  possible when imagining the placed objects as this creates more hooks for
  later revoke of the items. For the following recall one needs only to do a
  virtual walk through the same trajectory and to recollect the items from
  their places (loci). A vivid imagination is always beneficial for these
  mental exercises.

  %Laura's subject with super memory, having a condition of astasia? has been implementing a similar method, probably unconsciously.. (or a paragraph down?)
  %every super memory human tested so far has been using some memory technique that relies on methos of loci or story
  %creation. Even Luria's subject...colorful memory
  %sequences, tables that could be read by raws, columns and diagonals; after 16 years still remembers
  %using streets from his home town or moscow to place items that has be remembered when given very long lists
  %very strong imagination,  severe synestisia where he feels the color, shape, weight, touch, etc of every word he hears or reads..certain problems in concntration while reading and especially while eating..
  %He never changed items with similar items or with synonims, however, during recall often he could skip some item that was placed in a `dark place' or was not very well distinguished form the background..
  %weak in the logical `organisation of memorizing'
  %bad memory on faces: they are so changeable, changing colors all the time, depending on the mood of the person, on the moment of meet..

  %It is, however, an open question whether the sequential arrangement is an
  %universal quality, or whether some more complex schemata can be used for
  %memory formation and recall. For example, can a two-dimensional pattern be
  %used as a base for the storage of new memories?

  Since ancient times it is known that memory can greatly benefit from
  predefined sequences. All known mnemonic techniques based on the method of
  loci or the story method (e.g., creating a story with the list items to be
  remembered) rely on the sequential storage and the recall of associated
  memories.

  Sequences have been proposed also as a more general model for the occurrence
  and development of the mental processes. The oldest known references to the
  concept of association of thoughts and ideas can be tracked to Plato and
  Aristotle from the ancient world \citep{Plato:Phaedo, Bloch2007}. Later, in
  the western world the thought sequence concept was further developed by early
  modern philosophers \citep{Hobbes, Locke, Hume, Hume2, Stewart}. David
  Hartley, a founder of the associationist school of psychology in the $18^{\rm
  th}$ century, proposed that even the most complex thought processes can be
  explained as sequential activations of clusters of elementary senses and
  representations. At the turn between the $19^{\rm th}$ and the $20^{\rm th}$
  century, a lot of work devoted to associations was contributed in the field
  of psychology: The thought experiment about William James' bear attempts to
  explain the physiological response of the body as sequence of mental events
  after the sensory arousal of seeing a bear \citep{James1884}.
  \cite{Pavlov1897} published his seminal work on the conditioned reflex, where
  he demonstrated the physiological effects of classical conditioning. Further
  work in experimental psychology conceived that sequences of experiences and
  their representations are in the base of motor action, cognition, and
  judgment \citep{Watt1904, Titchener1905, Washburn1916}.

  Various disciplines have looked at the sequences of motor actions, thoughts,
  and memories as a basic behaviour substrate of the human nature. But how are
  such sequences represented in the brain, what is their neural basis?

    %Cajal: neuron doctrine, neurons have 3 parts and propagate impulses only in
    %one direction from one neuron to other; bold assumtption from his side
    %based solely on anatomical observations... ref textura del sistema nerviso... and linas2003, Nature 

    %- from psychology(washburn watson) to neurophysiologistb (lashley, in The problem of serial order in behavior) talking about hierarchical organization of plans, and chunking..
    %Ironically, Lashley's student, the founder of CNS propoes PSs..

\section{Hebbian theory}
\label{sec:hebb}
  In the 1940s, Donald Hebb, a Canadian psychologist and neurophysiologist,
  stated that the problem of understanding behavior is the problem of
  understanding the total action of the nervous system and vice versa. He was
  the first to apply the principles of the neuron doctrine \citep{Cajal1894} in
  a coherent framework in an attempt to explain the mechanisms behind the
  thought processes. In his seminal work, \cite{Hebb49} introduced three
  concepts that are widely used nowadays in the neuroscience community: a
  learning rule now known as `Hebbian learning', `cell assemblies', and `phase
  sequences'. Hebb suggested that neurons that if a neuron is participating in
  the firing of another neurons, then the corresponding synaptic efficacy is
  increased, or a new synapse is created (Figure~\ref{fig:hebb}). Thus, neurons that fire together form
  strong connections and organize into assemblies representing abstract mental
  concepts, today known as `Hebbian assemblies', or simply assemblies. Once an
  assembly is activated, it can ignite associated concepts by activating the
  corresponding assemblies, thus forming a sequence of activation or a `phase
  sequence'.
   
    \begin{figure}
      \center
      \includegraphics[width= 25pc]{hebb2.png}
      \caption{
        The top panel shows sketches from \citep{Hebb49}. {\bf top-left:} Cell
        A excites cell C, which excites cell B. The synaptic efficacy from A to
        B will be potentiated, or new synapses will be created after repetitive
        activation of cells A and C. {\bf top-right:} Cells A, B, C receive
        converging inputs (not shown). Cells D, E, and X among many other cells
        have connections with A, B, and C, and thus, contribute to the
        integration of their activity. {\bf bottom:} Neural assemblies can be
        reliably evoked by thalamic inputs (left) or manifest themselves during
        spontaneous network activity (right). Gray dots present neuronal cells,
        and colored dots show the cells activated during 3 consecutive
        $300\,\rm ms$ time windows. Assemblies show reliable spatio-temporal
        activations where the sequential order of firing is preserved. The
        right panel shows an overlap of thalamic and spontaneous activations.
        Scale bar $50\, \mu m$ (\citealp{Luczak2012}; adapted with permission).
             }
    \label{fig:hebb}
    \end{figure}

  These concepts led to the development of `connectionism' as a movement in
  neuroscience, cognitive sciences, and philosophy. Hebb's theory had a huge
  influence on the early-day machine learning development, and in particular on
  the research in artificial neural networks. For example, the Hopfield
  network, a neural network model consisting of binary neurons as a content
  addressable memory model \citep{Hopfield1982} uses Hebbian learning as a rule
  to adjust connection weights. Another technique based on the Hebbian theory
  is Oja's rule for unsupervised learning, which can extract the main features
  (principal components) of datasets \citep{Oja1982}.

  In the following paragraphs I briefly review literature that was triggered by
  Hebb's theory. In particular, I focus on neurophysiological studies that shed
  some light on the learning processes and the representation and detection of
  cell assemblies in the brain.
  
  \subsection{Hebbian learning: fire together, wire together {\protect\footnote{A
  paraphrase from ``neurons wire together if they fire together'' \citep{Lowel1992}}}}

    The idea that the formation of new memories does not require new neurons is
    old. Already \cite{Cajal1894} suggested that for the creation or change of
    memories the brain might simply strengthen old synapses instead. This
    hypothesis stood the test of time and now is known as the synaptic
    plasticity and memory hypothesis \citep{Martin2000, Takeuchi2014}. Half a
    century after Ram\'{o}n y Cajal, \cite{Hebb49} postulated that if neuron A
    takes part in the firing of neuron B, then the efficacy of the synapses from
    neuron A to B is increased, or that even new synaptic connections could be
    formed. This learning rule is known nowadays as `Hebbian learning'.

    The first demonstration of the plastic nature of synapses was demonstrated
    in anesthetized rabbits by \cite{Lomo1966} who found that a high-frequency
    stimulation to the presynaptic fibers in the perforant pathway increases
    the postsynaptic potentials (PSPs) measured in the dentate gyrus. Moreover,
    these changes in the synaptic efficacies lasted for hours. The idea that
    this synaptic facilitation might depend on the precise timing of activation
    was proposed by \cite{Taylor1973} who stated that a presynaptic spike
    shortly before the postsynaptic activity would facilitate the synaptic
    efficacy. In a computational model, \cite{Gerstner1996} showed that a
    sub-millisecond plasticity rule depending on the exact timing of pre- and
    postsynaptic firing can indeed lead to a Hebbian learning. In the following
    year in experimental work, \cite{Markram1997} demonstrated that a synapse
    can be differentially up- or down-regulated depending on the precise time
    difference of the synaptic activation and the postsynaptic action
    potential. The spike-timing dependent plasticity (STDP) rule was described
    experimentally in a greater detail by \cite{Bi1998} who confirmed an
    asymmetric temporal window of plasticity between pyramidal neurons in
    hippocampal culture (see Figure~\ref{fig:stdp}, top-left panel). According to this
    rule, the order of spiking determines the sign and magnitude of synaptic
    change; a presynaptic spike followed closely by a postsynaptic one leads to
    long-term synaptic potentiation, while a reversed firing order leads to
    long-term depression. The exact biological mechanism behind STDP remained
    elusive.

    \begin{figure}
      \center
      \includegraphics[width= 32pc]{stdp.png}
      \caption{
        Spike-timing dependent plasticity relies on the precise time
        difference of pre- and post-synaptic activations. {\bf top-left:}
        Critical window for induction of synaptic changes (\citealp{Bi1998}; adapted with permission).
        Positive time differences $\Delta t >0$ (first pre- and
        then post-synaptic activation) lead to synaptic potentiation, and
        negative time differences lead to depression 30 minutes after a
        repetitive pairing protocol. {\bf top-right:} A symmetric STDP temporal
        window between CA3 pyramids {\it in vitro} \citep{Mishra2016}. {\bf
        bottom:} Various forms of STDP rules reported in the literature, and
        summarized by \cite{Feldman2012}.
             }
    \label{fig:stdp}
    \end{figure}

    The classical asymmetric exponential temporal window has been confirmed by
    a number of studies \citep[e.g.,][]{Debanne1998, Zhang1998}. However,
    because of the used techniques, the results are met with some reservations.
    For example, \cite{Lisman2005} point out that in the aforementioned
    studies, the postsynaptic firing was induced by a current injection instead
    of more natural synaptic inputs. Spikes evoked by the postsynaptic
    potentials alone at low frequency do not lead to any synaptic potentiation
    \citep{Wittenberg2006} suggesting that additional factors might be involved
    in the plasticity processes. The conventional STDP rule
    \citep[i.e.,][]{Bi1998, Kempter1999} is not universal for all synapses.
    Various STDP temporal windows (Figure~\ref{fig:stdp}, bottom panel) have
    been found in dependence upon the brain region, preparation, stimulation
    protocol, and the type of pre/post-synaptic neurons \citep{Feldman2012,
    Vogels2013}. Recently, \cite{Mishra2016} demonstrated that in hippocampal
    slices of mature rats, the potentiation of CA3-CA3 recurrent excitatory
    synapses is independent on the temporal order of stimulation, resulting in
    a symmetric STDP curve. The authors argue that the symmetric STDP curve
    (see Figure~\ref{fig:stdp}, top-right panel) allows for a reliable storage
    in the associative CA3 network. {\it In-vivo} experiments point out that
    the sign of synaptic changes might also depend on the exact phase of
    stimulation during the hippocampal theta oscillation \citep{Hoelscher1997}.

    The term `Hebbian learning' is widely used nowadays in the theoretical and
    physiological literature. While such plasticity rule is shown to exist in
    various brain areas, it is not universal. Depending on the behavior state,
    the local network oscillation phase, or on the type of synapse, various
    plasticity rules shape the connectivity matrix of the brain.

  \subsection{Hebbian assemblies.}
    The idea that a memory, or an `engram' is stored into a single cell or a
    group of neural cells was proposed by \cite{Semon1904}. According to Semon,
    engrams are representations of specific stimuli, and engram complexes are
    the basis of memory traces. Later, \cite{Hebb49} defined the assembly
    concept more specifically by postulating that neurons receiving similar
    inputs form strong connections among themselves, organize into assemblies
    or engrams. Neurons in these assemblies are activated synchronously when
    the associate mental concept is revoked. While not strictly defined, the
    cell assembly has been a well-accepted conceptual tool that is widely used
    in theory and experiments.
   
    The search of the assembly representation in the brain has proven to be a
    challenging task. Many questions are still open: What what is the size of
    an assembly? How reliable is the participation of neurons from one
    activation to another? How stable are the assemblies during reactivation?
    Do assemblies overlap? Last but not least, is the assembly concept too
    general and thus too difficult to prove and disprove \citep{Wallace2010}?
    The most promising hints for the existence of assemblies come from the
    hippocampus and the early processing sensory brain areas where
    experimentators have better control over the inputs in comparison with
    other `higher brain areas'. Few lines of evidence, such as connectivity
    patterns between neurons and signatures of activity in the local circuits,
    suggest for the existence of neural assemblies.
    
    %: some cells fire
    %specifically to certain sensory inputs or represent some more complex
    %mental concepts (e.g., grandmother cells \citep{}, place cells); clusters of
    %connectivity have been reported in various brain regions where cells that
    %fire together share more common neighbours than expected by chance,
    %neuronal activity can transiently and abruptly switch from one activity
    %mode to another as sign of discrete population activity \cite{Hopfield}.
    %as memories change and update..., and so do dendrites \cite{Yang2009}

    When studying neural networks, an apparent problem is the complexity of the
    circuits. {\it In-vitro} experiments deal with minimal networks consisting
    of tens of thousands to hundreds of thousands neurons, and the number of
    possible connections scales as the square of the network size. Despite the
    rapid development of the current technologies, mapping such connectivity is
    not possible yet. Therefore in their models, theoreticians often flavor the
    random networks as a tool to address complex network connectivities.
    However, the appearance of random and independent connections in biological
    neural networks is scarce. Clustered connectivity patterns have been
    reported in various cortical areas \citep{Song2005, Ko2011, Perin2011,
    Shimono2015} as well as in the hippocampus \citep{Takahashi2010,
    Guzman2016}. Moreover, neurons form bidirectional connections more often
    than expected by chance \citep{Markram1997, Song2005, Takahashi2010,
    Ko2011, Perin2011}. The distribution of synaptic weights is non-uniform.
    Local neocortical networks exhibit distributions of synaptic weights that
    are heavily skewed, and bidirectional connections are stronger than
    uni-directional connections \citep{Markram1997, Song2005, Buzsaki2014}.
    \cite{Shimono2015} have shown that in neuron culture, the microconnectome
    has different levels of clustering, from a few neurons up to hundreds of
    neurons. The authors suggested that the different levels of organisation
    lead to different levels of robustness, where larger clusters are more
    robust to be activated against noise such as errors in synaptic
    transmission or external noise.

    How is this nonrandom connectivity related to the circuit activity? It has
    been shown that ongoing spontaneous activity in the cat visual cortex
    switches between different states some of which correspond to the
    orientation maps of neurons \citep{Kenet2003}. These results suggest that
    neurons' preferred tuning is not purely due to the sensory input but also
    reflects some intrinsic network structure. Comparison between spontaneous
    and evoked activity in the auditory and somatosensory cortices of rats
    showed that network dynamics is largely conserved between states, and that
    activity was drawn from a rather limited `vocabulary' \citep{Luczak2009,
    Luczak2012}.
    
    Although the direct link between connectivity and activity is rather
    sparse, there are some evidences that neurons in functional assemblies
    (groups of neurons that are co-activated simultaneously) are highly
    connected. \cite{Takahashi2010} has shown that in organotopic slices
    neurons that exhibit highly correlated activity are connected with higher
    probability than uncorrelated neurons. Moreover, connected neurons share
    common input and output neurons more than expected by chance. {\it In-vivo}
    work by \cite{Ko2011} in layer 2/3 of the mouse primary visual cortex
    revealed that cells with similar preference have bigger probability of
    uni-and-bi directional connections. Moreover, \cite{Cossell2015} showed
    that the synaptic strength between these neurons varies over 2 orders of
    magnitude, and the strongest connections are between neurons with very
    correlated responses, while weak synapses are between neurons with
    uncorrelated responses. Although hugely outnumbered, the strong inputs
    disproportionally control the response of neurons. 

    A number of evidences point out that fine-scaled subnetworks (from tens up
    to hundreds of neurons) specialise in processing similar information where
    correlation in activity is correlated with connectivity. While the
    assemblies in the primary cortical areas (V1, A1, etc.) are constituted by
    neurons that are spatially close to each other, and thus, form `receptive
    maps' \cite[e.g.,][]{Bathellier2012, Cossell2015}, in the hippocampus the
    assemblies are spatially distributed \citep{Guzman2016}.

    Another feature of Hebbian assemblies predicted by the theory is the
    discrete activation of neural populations as the network activity gets into
    various attractor states \citep{Hopfield1982}. Attractors are defined as
    all-or-none states in the network activity, where states close to these
    points are attracted to them. Such attractor-like activity that exhibits
    reliably revoked spatio-temporal patterns have been observed in slices
    \cite[e.g.,][]{Cossart2003, MacLean2005}. More recently,
    \cite{Bathellier2012} have shown that sound stimuli are evoking
    attractor-like dynamics in superficial layers of the auditory cortex with
    an abrupt switching between the different discrete modes. The discrete
    modes constitute of partially overlapping subpopulations where the same
    neurons can take part in a few assemblies, and assemblies interact in a
    competitive fashion. Due to the complexity of stimuli representation, such
    measures are more challenging in the higher cortices. By projecting the
    measured neuronal activity in high-dimensional state space by kernel
    methods (PCA), \cite{Balaguer2011} have shown that on-going activity in the
    higher cortices posses attractor-like dynamics. Abrupt transitions between
    attractor states have been also reported in experiments where the external
    sensory cues were gradually changed. For example, continuously varying an
    odor results in abrupt changes in the odor representation in the olfactory
    bulb of zebrafish \citep{Niessing2010}. Or changing gradually the
    environment evokes place cells abruptly and simultaneously to change
    representation \citep{Wills2005}. As shown in virtual teleportation
    experiments, one cycle of the theta oscillation is a temporal unit for
    expressing an attractor state in the hippocampus \citep{Jezek2011}. It is
    not known yet whether this discretization of activity is due to purely
    internal dynamics or the inputs to the hippocampus are already discrete
    with theta-cycle resolution. 

    %Okun2012: neuronal ensembles change as the network state changes
    %Rothschield: A1, functional organazation of neurons into subnetowrks with some discrete activty

    %Both the hippocampus and the primary cortical areas seems to be populated with competitive assemblies \cite{?}.

    A lot of work has been devoted into capturing this `holy grail' in
    neuroscience called neural assemblies. Multiple experimental evidences hint
    that the brain networks are indeed populated with clusters of neurons that
    share common representations. However, we do not have a clear picture of a
    more general syntax explaining how these subnetworks can interact,
    especially in higher-brain areas. Relaxing the definition of a cell
    assembly, \cite{Buzsaki2010} suggests that hierarchical organization of
    cell assemblies may constitute syntactical rules that define first-order
    and higher-order relationship. Cell assemblies are then defined not by
    connectivity but by their synchronous activation during a concept
    representation. \cite{Pulvermuller2010} argues that the assembly syntax
    might be tightly related to the linguistic syntax that we use. When
    introducing the assembly concept, \cite{Hebb49} also proposed that
    assemblies are organized in phase (or assembly) sequences, and these
    sequences themselves are evoked in a sequence of activations.

  %neural syntax facilitates the interaction between different hierarchies of assembly sequences..

  %Harris et al: theory of harries neocortex probing for actiavations?
 
  %possible pluses: reliability, more reliable activity trasmission than single neurons/readers as tolerates variability in single neuron firing; possibly more adaptable in learning new concepts and keep the old ones

  %Jensen and Lisman96, Buzsaki2010 neural words might be short assembly sequences that fire in single gamma cycles..
  %assemblies, physiology, fmri?
  
    %%%%%%kkk

    %perin2011: lego-blocks fro assemblies...
    %with similar receptive fields form assemblies that 
    %can be activated either spontanouesly or by weak input stimulus 

    %Miyashita88b: neurons fire specifically to objects in neocortex of 
    %there should be older studies on that!!

    %???
    %Liu2012: fear memory actiavtion through optogenetics
    %also others that did that???

    % maybe that should go at the end of this subsubsubsusbsubsusbsusbsusbsusbsubsubsubusbsubsubsection
  %A lot of work has been devoted into capturing this `holly grail' in
  %neuroscience. While numerous evidence point out that neurons might indeed
  %organize into functional assemblies, even more questions about the operation
  %of the nervous function arise.

  %miyashita88a: fractal recognition neurons in monkeys
  %we become what we have learned to recognize, 

  %experiment to try with more common common and uncommon input and see how neurons respond
  %prediction: high firing for common and low for uncommon (unnatural)
  %fractals are probably some of the most natural?
  
  %some stability stuff?
  %fernando-ruiz: functional assemblies in CA3 in-vivo; changing participants..

  %Question is why does the old state still manifistate itself after the mouse is the new environment, is this the left-over of some working memory, what is the basis of it. Is it from hippocampus or from external working mewmory input? Jezek2011 teleportation


%A big question is how stable are the representations of stuff in brain, and if they change how they change 
%as a result of experience; consolidation of memories/associations, how does it change the representation of these objects, in hippocampus or in other regions of neocortex???


  %refs
  %H Markram,  Physiology and anatomy of synaptic connections between thick tufted pyramidal neurones in the developing rat neocortex.
 
  %Semon, R. (1904). Die Mneme [English Translation: The Memory]. Leipzig:
%Wilhelm Engelmann.

  \subsection{Phase sequences}
    The third, less known principle proposed by \cite{Hebb49} is about `phase
    sequences': once an assembly is activated, i.e., the underling neurons are
    firing, its activation would propagate and activate another assembly, thus
    mental concepts would ignite associated concepts. Hebb suggested that such
    sequential activation of assemblies of neurons underlies our most complex
    mental processes. Without going into much details, he suggested that phase
    sequences can interact with each other and organize more complex
    hierarchies and sequences. 

    Various brain areas reveal spatiotemporal activity patterns that repeat
    over time \cite[e.g.,][]{Wilson1994, Kenet2003, Berkes2011}. Therefore, it
    is common to find groups of neurons that fire reliably in a temporal order
    during the execution of the temporal patterns. Moreover, neurons do not
    always fire in a single burst but show more various patterns such as ramped
    activity, or more complex or chaotic firing. The microcircuits
    underlying such activity are largely unknown, and there are scarce
    evidences pointing to a sequential activation of neural assemblies. An
    interesting idea by \cite{Goldman2009} states that decomposition the
    connectivity matrix of a recurrent network in Schur modes allows to project
    the activity as a feedforward interaction between the activity modes. Thus,
    with the appropriate projection, any recurrent network can be viewed as a
    `feedforward network in disguise'.
    %Howevwe FF nets are powerful model as their relativily simplicirty compared to RNN
    %and are largely applied in artificial nets : MLP, DL, ?. while RNns are sparsely applied
    %Hopfield, LSM, ESN. FF are easy to grasp while RNN are too complex.. Goldman here?

    Some of the most striking examples of precise sequences of neural activity
    come from songbirds (Figure~\ref{fig:songbird}). Male songbirds such as the zebra finch perform complex
    stereotypical songs consisting of variable patterns on multiple timescales.
    The songs are learned from a tutor bird and are extensively rehearsed for
    around 60 days until deployed in practice upon reaching sexual maturity
    \citep{George1995, Doupe2004}. It is believed that song syllables and tempo
    motifs are stored in a forebrain nucleus called HVC (formerly, an
    abbreviation of the now invalidated region Hyperstriatum Ventrale pars
    Caudalis, now HVC is a stand-alone name) where neurons are firing
    selectively to sounds, syllables, or sequence of syllables \citep{Yu1996}.
    The HVC premotor neurons are shown to fire sparsely and very precisely in
    stereotypical sequences of bursts with very small jitter ($<1\,\rm ms$)
    when aligned with the performed song \citep{Hahnloser2002}. The same
    sequences of activity are replayed offline during sleep, which is believed
    to support the memory consolidation of songs \citep{Dave2000}. While the
    connectivity within the HVC is mostly unknown \citep{Hamaguchi2012,
    Poole2012}, most theoretical studies focus on models of feedforward
    networks for learning and performing songs \citep[e.g.,][]{Li2006,
    Long2010, Hanuschkin2011}. Experiments with lesions have supported the
    feedforward network model \citep{Poole2012} Moreover, \cite{Kosche2015}
    have shown that the generation of song sequences relies not only on
    excitation, but also patterned inhibition. The circuit generating these
    patterned sequences remains, however, illusive.  

    \begin{figure}
      \center
      \includegraphics[width= 32pc]{songbird.png}
      \caption{Firing sequences in songbirds. Spectrogram (top) and acoustic
        signal (the black curve below) of a song motif. Spike-raster plot of
        ten HVC neurons and two HVC interneurons recorded in one bird during
        singing (colored ticks). Each row of ticks marks spikes generated
        during one rendition of the song; roughly ten renditions are shown for
        each neuron. HVC neurons burst reliably at a single precise time in the
        song or call; however, interneurons spike or burst densely throughout
        the vocalizations (\citealp{Hahnloser2002}; adapted with permission).
             }
    \label{fig:songbird}
    \end{figure}

    For the assembly sequence hypothesis, it is important that the activity of
    one assembly synchronously drives the activity in another assembly. Such
    coding is compatible with the temporal code where the exact spike timing in
    relation to other neurons is crucial for the information transmission.
    Indeed, sequences of neural activity on fast, millisecond time scales have
    been reported in various neuronal circuits such as, just to name a few,
    insects \citep{Fushiki2016}, fishes \citep{Romano2015}, dragons
    \citep{Shein2016}, in mammalian cortex \citep{Euston2007} and hippocampus
    \citep{Lee2002}, as well as {\it in-vitro} preparations \citep{Mao2001,
    Segev2004, MacLean2005, Kruskal2013}. Sequence replays depend on the
    behavior state \citep{Almeida2014}, and reverberations are enhanced mostly
    in the desynchronized states \citep{Contreras2013, Buzsaki1983} when the
    local circuit activity is intrinsically driven. 

    In the mammalian brain, the hippocampus exhibits a particularly rich
    repertoire of sequential activations of neurons. Firing patterns of
    behavioural sequences are replayed on various time scales, e.g., behaviour
    time scales up to minutes during REM sleep \citep{Louie2001}, compressed
    sequences of tens of milliseconds in exploratory `theta sequences'
    \citep{Skaggs1996, Dragoi2006, Gupta2012, Feng2015}, and a few millisecond
    precision of sequential firing during slow-wave sleep and resting
    \citep{Lee2002}. Moreover, depending on the behaviour state, the
    hippocampal replays can exhibit firing sequence in the same or reversed
    temporal order as experienced in the awake state \citep{Foster2006}.

\section{Hippocampus: a brief survey}
\label{sec:hippo}
  Time is so intimate for us that it is difficult to imagine to live without
  it. We live in a dynamical system that constantly changes the environment
  that we perceive, and so we change as well. As our changes reflect past
  events, they can be generalized as memories. There are memories at various
  levels, such as physical, genetic, neurophysiological, etc. When we use the
  word `memory' in everyday talk, usually we refer particularly to the
  declarative memory that includes episodic (spatial and autobiographical
  experiences and events) and semantic (vocabulary, facts, concepts, etc.)
  types of memories.  In humans and other mammals, the hippocampus is a vital
  brain structure for storing the past in the form of declarative memories and
  planning for the future.
  %Hp is a remarkable structure that supports the formation of new declarative memories.

  The aim of this section is to give the reader a basic picture of the
  hippocampus. I start with a brief description of the hippocampus anatomy and
  then dive into its functional role.

  \subsection{Anatomy}
    The hippocampus is first described by Julius Caesar Aranzi in 1564 and is
    named due to its resemblance to the seahorse (from Greek $\iota \pi \pi o
    \kappa \alpha \mu \pi o \varsigma$: seahorse, see Figure~\ref{fig:hp}). The
    hippocampus is a subcortical structure, part of the limbic system. The
    hippocampal formation is remarkably preserved across mammalian species
    \citep{Manns2006, Clark2013}. Mammals have 2 hippocampi located bilaterally
    in the medial temporal lobes wrapped by the cerebral cortex
    (Figure~\ref{fig:hp}). The hippocampus proper is one of the most
    extensively studied regions in the mammalian brain in terms of anatomy,
    electrophysiology, and behavioral function.

    \begin{figure}
      \center
      \includegraphics[width= 25pc]{hippo.png}
      \caption{
        {\bf A.} Location of the hippocampus (red color) the human brain
        (transparent yellow) viewed from the side (left panel) and the back
        (right panel).
        {\bf B.} A comparison between a human hippocampus and a seahorse.
        {\bf C.} A drawing of the hippocampal formation after a horizontal cut.
        A few principal neurons for each region are sketched. The panel inset
        illustrates the major excitatory pathways.
        {\bf D.} A schematic of the spatial relations between neurons within
        the CA1 region and major inputs/outputs from other areas. The pyramidal
        neurons and a few classes of interneurons are sketched according to
        their laminar location. The layers are color coded in the right-hand
        side of the panel.
        Attributions:
        A is adapted from lifesciencedb.jp (Creative common licence). B and C
        are adapted from Wikipedia (Creative Commons licence). C is a modified
        after an original drawing of \cite{Cajal1911}. D is adapted from
        \cite{Somogyi2014}.
             }
    \label{fig:hp}
    \end{figure}

    The hippocampal formation consists of the hippocampus itself, the
    subiculum proper, and the entorhinal cortex,
    which provides the major input and is the main output of the hippocampus.
    The hippocampus can be roughly divided in dentate gyrus, and the {\it cornu
    ammonis} regions (namely, CA3, CA2, and CA1). The hippocampus has gained
    popularity in neurophysiology partly because of the unidirectional
    transmission of information from dentate gyrus to the CA regions, which
    facilitates the study of the local microcircuits. The main input to the
    hippocampus is provided by the entorhinal cortex (EC). Activity from the
    superficial layers of the EC propagates through the entorhinal-hippocampal
    (or trisynaptic) loop and projects back to the deep layers of the EC
    (Figure~\ref{fig:hp}, specify where!!). Much like the cortex, the
    hippocampal subregions are organized in well-defined laminar structures
    where different layers are populated by different cell types, and the
    different synaptic inputs are confined to specific layers. 

    \textbf{Dentate gyrus microcircuit.}\,
    The dentate gyrus (DG) consists of three neuronal layers: molecular,
    granular, and polymorphic. The principal excitatory cells, the granule
    cells are located in the granular layer. The main input to the DG comes
    through the perforant path from layer II of the entorhinal cortex
    \citep{Squire1992} while the output consist of unidirectional connections
    to the CA3 region through the mossy fiber (MF) synapses. Due to their
    extremely high efficacies where a single presynaptic spike can drive the
    postsynaptic neurons to fire, the MF synapses are also known as
    `detonators' \citep{Bischofberger2006}.

    Another peculiar feature of the DG is the fact that granule cells are
    generated throughout the whole life in rodents as well as in humans. Due to
    the neurogenesis, the number of neurons in DG can vary, and subjects that
    use more extensively spatial navigation tend to have a larger number of
    granule cells. Moreover, the size of the DG correlates with spatial memory
    \citep{Maguire2000, Spalding2013}.

    \textbf{CA3 microcircuit.}\,
    The CA3 hippocampal area is an important focus of this thesis. Like the
    rest of the hippocampal subregions, CA3 possesses a well-defined laminar
    structure that confines the localization of the different neuron types and
    inputs (see Figure~\ref{fig:hp}). The deepest layer of the CA regions is
    the stratum lacunosum moleculare (sl-m), followed the by stratum radiatum (sr), and the
    pyramidal cell layer (pcl). The most superficial layer is the stratum
    oriens (so). All layers are populated by inhibitory interneurons. There are
    over 20 types of interneurons that can be classified depending on the
    location, morphology and genetic expression profile \citep{Maccaferri2003,
    Klausberger2008}. The principal cells, the pyramidal neurons are located in
    the pyramidal layer and comprise around $90\%$ of all the cells in that layer. Their
    number in rats is $\sim 3 \cdot 10^5$ \citep{Boss1985, Boss1987} in CA3,
    while the human hippocampus has $\sim 2.7 \cdot 10^6$ neurons in CA3.
    Pyramidal neurons extend their basal dendritic trees in the superficial
    layers, while the apical branches are in the deep cell layers. The axons of
    pyramidal neurons can innervate around 70\% of the dorsal-ventral axis of
    the hippocampus and project to other pyramids and interneurons in the
    stratum radiatum and oriens \citep{Sik1993, Li1994}.

    A major input to the CA3 comes through the MF synapses from the dentate
    gyrus. The MF terminals form a distinguishable layer within the stratum
    radiatum, sometimes referred to as stratum lucidum. Two different major
    inputs come from the entorhinal cortex, layer II: a direct projection from
    the axons of the EC layer II pyramidal neurons reach the CA3 region; and an
    indirect input from EC layer II stellate cells projecting to the DG through
    the perforant path, and then through the MF synapses to the CA3
    \citep{Tang2014}. 
    
    The CA3 hippocampal area was considered to be densely connected 
    \cite[$\sim3 \%
    $ in][]{Miles1986} relatively to the CA1, but more recent evidences show that
    the connectivity between principal cells is around 1\% \citep{Guzman2016}.
    The major output of CA3 is the excitatory axonal projections to the
    ipsilateral and contralateral CA1 area via the Schaffer collaterals
    \citep{Finnerty1993}. The reader is kindly referred to \cite{Duigou2014} for a
    more detailed description and analysis of the connectivity within the
    CA3 region.
    %plastic conenctivity who????

    \textbf{CA2 microcircuit.}\, Probably because of its smaller size, the
    hippocampal CA2 area has received a negligible amount of attention compared
    to its neighbors CA1 and CA3. The area gets projections from EC layer II
    pyramids through the perforant path, but no mossy fiber inputs.
    Interestingly, CA2 in mice is involved in sociocognitive memory processing
    \citep{Hitti2014}.

    \textbf{CA1 microcircuit.}\,
    The CA1 hippocampal region has a similar laminar structure as CA3.
    However, the DG input is absent and the entorhinal input originates from
    EC, layer III. The principal neurons are more densely packed than in CA3,
    and there are around $3.5\cdot 10^5$ neurons in rats \citep{West1991} and
    $20.9 \cdot 10^6$ in humans \citep{Simic1997}. For a more in-depth analysis
    of the recent findings on anatomy and connectivity of different cell types
    in the CA1 region, I kindly refer the reader to \cite{Bezaire2013}.

    \textbf{Subiculum proper.}\,
    The region is subdivided into parasubiculum, presubiculum, postsubiculum,
    and prosubiculum. The subiculum is the major output of the hippocampus and
    receives projections from the CA1 region and EC layer III neurons.

    \textbf{Entorhinal cortex.}\,
    The entorhinal cortex provides the major interface of other neocortical
    areas to the hippocampus. The EC is part of the neocortex with the typical
    6-layered structure. Principal neurons are the stellate cells which
    populate layer II, and the pyramidal neurons, which populate layers II,
    III, and V.  As mentioned above, principal cells from layer II of EC are
    projecting to the dentate gyrus and CA3, while layer III neurons are
    innervating the CA1 region and the subiculum. Inputs from the hippocampus
    are impinging on the deeper layer V of the EC.
    %mcNaughton2006????

    %1 DG GC to CA3: 14 synapses; 1 CA3 PC gets 46 inputs from DG GC to CA3 Amaral1990
    %12000 recurrent CA3 synapses Amaral1990
    %3750 PP inputs Amaral90

  \subsection{The role of the hippocampus in the human brain}
    In the past, the hippocampus was assumed to be involved in various tasks. A
    dominant hypothesis until the end of 1930s suggested that its involvement
    in olfactory processing \citep{Andersen2007}. Later hypotheses assigned
    roles in emotions and attention. Some earlier results from Brown and
    Sch\"{a}fer in 1888 and Bekhterev in 1900 hinted to the involvement of the
    hippocampus in memory \citep{Andersen2007}, but it was not a popular view
    until \cite{Scoville1957} presented a study on the amnesic patient H.M.

    Henry Gustav Molaison (\textborn~26.02.1926, \textdagger~2.12.2008), known
    as patient H.M. during most of his life, is probably the most influential
    patient in neuroscience. In an attempt to be cured from epileptic seizures,
    he underwent a brain surgery in 1953 in which most of his hippocampi,
    amygdalae and the surrounding entorhinal and parahippocampal cortices were
    removed. The operation was successful in its initial goal that he no longer
    suffered from seizures, however, at the price of some `side effects'. After
    the surgery H.M. was diagnosed with anterograde amnesia, i.e., the
    inability to form new (declarative) memories. Moreover, he suffered from
    temporally graded retrograde amnesia, i.e., he could not or had
    difficulties to recall events that took place shortly before the surgery
    while more distant memories from the past were intact. While subjected to
    numerous psychological experiments until his death in 2008, H.M. showed
    intact intellectual abilities and short-term memory. He revealed that the
    memory system is not a single entity but that there are multiple subsystems
    with possibly different neural origins. For instance, he could acquire new
    implicit memories when subjected to repetition priming \citep{Corkin2002}.
    Moreover, he showed intact motor learning skills in tasks that take
    multiple repetitions to be mastered. In a motor learning experiment, he could
    learn to draw seeing only a mirror reflection of his artistic piece, while
    not possessing any knowledge and memory that he is able to perform the task
    \citep{Corkin1968}. Studied for more than 50 years, H.M. pointed out the
    importance of the hippocampus for the formation of new episodic and
    semantic memories.

    Work with other hippocampal amnesic patients has revealed some
    dissociation between the two types of declarative memories, i.e., episodic
    (or autobiographical) and semantic. While the overall recall of distant
    memories is intact in amnesic patients, the recall of the autobiographical
    memories are somewhat not as vivid and detailed as in control subjects.
    \cite{Moscovitch2005} suggested that the hippocampus is actively engaged
    during the recollection of episodic memories, and possibly during the
    recall of semantic memories that have some autobiographical elements.

    The disfunction of the hippocampal formation causes not only impairment of
    remembering the past but also the inability to imagine the future. Work with
    amnesic patients with bilateral hippocampal damage has suggested the
    involvement of the hippocampus in the process of planning and imagining new
    experiences \citep{Klein2002, Hassabis2007}. Overall, these results give
    the hippocampus a very peculiar function as a `time machine' in our
    cognition. Apart from simply storing experiences and semantic knowledge,
    the hippocampus also lets us mentally explore past or future experiences.
    Such view naturally evokes the question whether H.M. or other amnesic
    patients ever experience mind-wandering.

    %one-shot learning: nakazawa2003; feng2015
    %involvement in spatial orientation..hp lesion leads to spatial disorientation (few interesting refs from jorge?)
    
  \subsection{Neural correlates of behavior} 
    After H.M. has sparked the interest to the human hippocampus, numerous
    electrophysiological studies have been conducted in animals in the
    search of the neuronal basis of memory.

    \cite{OKeefe1971} for the first time showed that some hippocampal pyramidal
    neurons in behaving rats are activated while the animal is at a specific
    location. As such cells code for the position in space, later they were
    named `place cells'. Following experiments have shown that a subset of
    hippocampal and entorhinal neurons can code for head direction
    \citep{Taube1990a, Taube1990b}, odors \citep{Wood1999}, environment
    boundaries \citep{Jeffery2006}, particular objects \citep{Manns2009}. The
    discovery of cells in the entorhinal cortex that fire at multiple locations
    in an environment and forming a particular hexagonal grid
    \citep{Hafting2005} have further attracted attention. In a very recent reports, \cite{Sarel2017} described
    hippocampal cells in bats that code for the vectorial representation of
    goal location, a firing pattern predicted previously by theory
    \citep{Stemmler2015}. So far, the rodent hippocampal formation has shown a remarkable
    repertoire of neural codes for spatial navigation that does not simply
    reflect the sensory input but reveals a more abstract representation of the
    environment.
  
    There is evidence for the existence of place cells and grid cells
    also in humans exploring virtual environments \citep{Ekstrom2003,
    Jacobs2013}. Experiments with monkeys have revealed the existence
    view-responsive neurons, that is, place-like and grid-like activity of
    neurons is evoked when the animal is looking at a particular location in
    the visual field \citep{Rolls1995, Killian2012}. This data suggests that
    hippocampal activity during navigation does not code only the current
    position of the subject but can also encode virtual exploration of
    environment.
    
    Recordings from the hippocampal formation of epileptic patients have shown
    that some neurons fire very selectively to particular objects, landmarks or
    individuals \citep{Quiroga2005}. The firing of these cells is strikingly
    invariant as, for example, they are activated when various photos of a
    Jennifer Aniston are presented, irrespectively of the orientation and the
    angle of shooting. Moreover, the corresponding cells are even activated
    when only the written name is presented to the subject. Such cells are
    called `grandmother neurons' and are unofficially known as `Jennifer
    Aniston neurons'.

    While the aforementioned studies show strong correlations between
    hippocampal activity and behavior, what is the role of such representations
    for the cognition, and for the memory in particular? With the advance of
    the neuroimaging techniques, and the optogenetics specifically, now
    researchers are able to track the activity of large population of neurons
    (e.g., thousands of cells) and manipulate their behavior during
    experiments. Tagging active cells during an association learning task, one
    can isolate the neural assembly that is activated during the learning. A
    following activation or inhibition of the assembly can evoke or suppress
    the learned response \citep{Cowansage2014, Tanaka2014}. These results
    provide an evidence for the active role of the hippocampus in the neural
    representations during behaviour.

  \subsection{Oscillations}
    % make a nice intro; check buzsaki's rhythms of braino
    Rhythms are omnipresent in the brain. Oscillations with various frequencies
    can be detected using extracellular electrodes or electroencephalography
    (EEG) in any brain region. These oscillations reflect the underlying
    neuronal activity that is synchronized across large populations of cells
    and are often strongly correlated to specific behaviors that the subject is
    engaged in.

    % oscillations in hp
    The hippocampus, particularly, shows some of the most striking rhythms in
    the mammalian brain. Electrophysiological recordings \textit{in vivo}
    reveal a rich repertoire of rhythms in the local field potentials that are
    state dependent. For example, the theta oscillation ($\sim 5-10 \rm\, Hz$)
    that is amongst the most regular rhythms is associated with active behavior
    such as walking and running, and also with the rapid-eye-movement (REM)
    phase of sleep \citep{Jung1938, Buzsaki2002}. Moreover the theta phase
    cycle modulates the amplitude of the faster gamma oscillation ($\sim 30-100
    \rm\, Hz$) that rides on top of it \citep{Bragin1995, Buzsaki2002}.
    Neurons' firing is largely modulated by the oscillation phase. While
    pyramids are mostly firing at the trough of the theta cycle, the various
    interneurons have different preferred phases \citep{Klausberger2008,
    Klausberger2009}. Moreover, neurons are differentially modulated by the
    gamma oscillation.

    \begin{figure}
      \center
      \includegraphics[width= 32pc]{hp_oscillate.png}
      \caption{
        {\bf A.} An illustration of a rat traversing a linear track (top panel). The
        firing place-field firing of a pyramidal neuron is colored coded
        (middle panel) with red denoting the location of maximum firing rate.
        Activity from a single run is shown in the bottom panel. The EEG activity
        (black curve) is dominated by the theta oscillation. The place cell is
        firing in bursts of spikes (red vertical bars) at higher than theta
        frequency causing each successive burst to occur at an earlier theta
        phase (\citealp{Huxter2003}; adapted with permission.
        {\bf B.} Bilateral coherence of theta and gamma oscillations
        (\citealp{Buzsaki2003}; adapted with permission).  Extracellular
        recordings from the pyramidal cell layer in the left (L) and right (R)
        hippocampi exhibit local field potential modulated by theta and gamma.
        Co-modulation of frequencies power in the two hemispheres (right
        panel). Theta power in one hemisphere is co-modulated with gamma power
        of the other hemisphere (yellow band at $9\,\rm Hz$ and 50--$100\, \rm
        Hz$, white arrowheads).
             }
    \label{fig:hp_oscillate}
    \end{figure}

    As mentioned above, during exploration pyramidal neurons are modulated by
    the current position of the animal, i.e., place cells. Interestingly, a
    place cell's firing pattern shows a peculiar modulations in respect to the
    local-field theta oscillation. When the animal is entering the
    corresponding place field, the neuron starts to fire at a particular phase
    of the theta cycle, the so called entering phase. As the animal progresses
    within the field, the neuron fires at earlier phases in each subsequent
    theta cycle. The range of phases that a neuron is precessing does not
    exceed the length of a theta cycle. This advance of firing phase in respect
    to the theta oscillation is known as `phase precession' \citep{OKeefe1993}.
    Thus, in addition to the information coded in the firing rates of neurons
    (i.e., the rate code), there is an additional information about the exact
    location of the animal carried in the exact spike timing in respect to the
    theta oscillation (i.e., the temporal code).
    
    Phase precession has been in the focus of research due to its assumed role
    in learning, and particularly, it has been suspected to be a mechanism for
    bridging the temporal gap between the behavior time scales (in the order of
    seconds) to the physiological time scales at which synaptic plasticity
    operates (order of milliseconds to tens of milliseconds) \citep{Bi1998}. For
    example, disruption of phase precession through a systemic administration
    of cannabinoid receptor agonist leaves the spatial selectivity of
    hippocampal neurons intact, but however, impairs the performance of rats on
    memory tasks \citep{Robbe2009}.
    
    When the animal traverses partially overlapping place fields, the
    corresponding place cells are sequentially activated. Moreover, as place
    cells show phase precession, the sequential spike order is preserved in the
    firing phase difference on a fine time scale (Figure~\ref{fig:hp_oscillate}).
    Moreover, the sequential spiking of neurons is more reliable than the
    spiking phase of single neurons \citep{Dragoi2006}. Such firing sequences
    within a theta cycle are called `theta sequences'. The temporal compression
    of sequences with in a theta cycle is assumed to be a crucial mechanism for
    the formation of memories \cite{Skaggs1996}, and in particular of the
    single-trial learning \citep{Rutishauser2006}. 
    
    What is the relation between phase precession and the theta sequence, and
    are they the two sides of the same phenomenon? \cite{Feng2015} observed
    phase precession in neurons during the first exposure of animals in novel
    environment, but however, reported the absence of theta sequences. From the
    second trial on, stable and reliable theta sequences were present. These
    result suggest that a single-trial experience is needed to form the
    sequence in the memory. Theta sequences, however, do not cover uniformly
    the whole environment, but are segmented in their representations.
    \cite{Gupta2012} showed that different theta sequences reliably represent
    chunks of the environment, frequently bounded by the space between
    landmarks.
    % looking in back in the past, or forward in the future

    Sequence replay is observed also during other prominent hippocampal
    oscillations, the sharp-wave ripples. Due to the central place of this
    phenomenon in this thesis, the following Section~\ref{sec:swr} is dedicated
    to their more detailed description.

\begin{comment}
  \subsection{Theories on the role of the hippocampus} 

    start with cognitive map?

    \citep{Marr1971} proposed a general theory for the function of the
    hippocampus that is still in the base of the majority of current
    hippocampal models. according to his model the hippocampus acts as a
    temporal storage for memories of events that are rapidly stored after a
    single-shot learning. later, a fraction of these memories are imprinted in
    the neocortex for long-term storage through a process called `memory
    consolidation'. moreover, he assigned different roles to the hippocampal
    subregions. for example, the dentate gyrus performs pattern separation,
    i.e., similar input patterns are matched to dissimilar output. on the other
    hand, the ca3 region performs the function of pattern completion, i.e.,
    recovery of a stored pattern given only a fragment of the input. while
    marr's theory is a rather theoretical study, \citep{buzsaki89} presented
    the `two-stage memory model' that matches better the electrophysiological
    phenomena found in the hippocampus. further theoretical models on memory
    consolidation with various degree of details and focus on the type of
    memories have been presented \cite{e.g.,}{}{squire92, rolls96,
    eichenbaum2004, jensen2005, hopfield2010, cheng2013b}
    tolman48: cognitive map

    fig from rolls96 (theory on hp f in memory)

    most (if not all) theories describing the function of the hippocampus on
    memory are founded on the synaptic plasticity and memory hypothesis
    \citep{Martin2000, Takeuchi2014}. an underlying theme in the majority of
    hippocampal models is the fact that the hippocampal formation is located on
    the top of a pyramidal hierarchy of information flow. this position
    provides the hippocampus with access to various cortical areas and with the
    opportunity to link neocortical representations from different modalities
    in single episodes. 

    to check:
    rolls and kesner 2006; hasselmo 2012; buzsaki2010

    In the brain however, there is a rich repertoire of rhythms and oscillations that synchronise
    distal brain areas.  theta-gamma nesting, time window of 20-30 ms, compatible with the epsp width.
    suggested that during a gamma cycle the activation of n unitary semantic block the assembly, and during theta
    a sentence consisting of ~7 words..

    %rutishauser2006: single-shot learning in hp

    %most theories rely on various neocortical representations that get
    %activates by hp simultaneously and thus get more connected

    %animals with hippocampal lesions perform on associations tasks where US and CS are 
    %simultaneously shown but fail when an interval separates both (refs in manns2005)
    %recognition memory also works for short periods but fails for longer periods..
\end{comment}

\section{Sharp-wave ripples}
\label{sec:swr}
  Sharp-wave ripples (SWRs) are hippocampal population bursts during which
  large numbers of neurons are orchestrated in precise firing. The sharp waves
  are reflected in brief (50--$100 \rm\,ms$) and large ($>1 \rm\, mV$)
  amplitude deflections observed in the local-field in the stratum radiatum of
  the CA regions (see Figure~\ref{fig:swr}). The incidence of SWRs ranges from
  0.01 up to $3 \rm \, events/sec$. The accompanying phenomenon of the sharp
  wave, the ripples are fast (120--$200 \rm\, Hz$) and short-lasting ($\sim
  50\, \rm ms$) oscillations that are most prominent in the pyramidal cell
  layer in the CA1 region. SWRs play an important role in memory processes such
  as learning, retrieval, consolidation, as well as planning.

    \begin{figure}
      \center
      \includegraphics[width= 32pc]{swr2.png}
      \caption{
        {\it A.} Forward and reverse replay of a place-cell sequence. The big central
        panel shows the local field potential (top), a raster plot of 13
        neurons where spike times are denoted with bars (middle), and the
        velocity of the rat in time (lower panel). Two $250\, \rm ms$ time windows magnify sequence
        replay during SWRs at the beginning and at the end of the
        lap (\citealp{Diba2007}; reproduced with permission).
        {\it B.} Local activation of PCs \textit{in vivo} evokes reliably SWRs
        with conserved waveforms (\citealp{Stark2014}; reproduced with
        permission). Depth profile of SWRs measure with multi-shank probes in a
        freely moving mouse (average from $n = 961$ events; vertical site
        separation: $100\,\rm mm$). Voltage traces (light gray gray) are
        superimposed on current-source density map. The black trace is from the
        site of maximum amplitude ripple (pyr: stratum radiatum); and thick
        gray trace is from the site of maximum amplitude SPW (radiatum).
        oriens: stratum oriens, lm: stratum lacunosum moleculare.
        {\it C.} \textit{In-vitro} SWRs can show regular occurrence
        (\citealp{Reichinnek2010}; reproduced with permission). SWRs are
        stable over the recording, as the same waveforms occur over hour-long
        intervals. Waveforms can be sorted according to their electrographic
        characteristics. Averaged waveforms show a congruent
        appearance (red and green) compared with averaged randomly chosen SWRs (black).
        %(\citealp{Schoeneber2014}, reproduced with permission).
             }
    \label{fig:swr}
    \end{figure}

  The sharp-wave phenomenon was first described in the rabbit hippocampus
  \citep{Stumpf1965}. Later reports come from dogs \citep{Yoshii1966}, monkeys
  \citep{Freemon1969}, humans \citep{Freemon1970}, and the first classification
  used the term `large irregular activity' \citep{Vanderwolf1969}.
  \cite{Buzsaki1992} first analysed and named the fast oscillation component,
  the ripple. Afterwards, sharp-wave ripples have been detected in every tested
  mammalian species. 

  In what follows, I am going to describe the sharp-wave ripples and some of
  their main properties. For the curious reader looking for a more in-depth
  information concerning SWRs, I highly recommend the recent \textit{opus
  magnum} by \cite{Buzsaki2015}.

  \subsection{Behavior correlates}
    {\it In-vivo} sharp-wave ripples (SWRs) can be observed during sleep and
    wakefulness. In the awake state, SWRs occur predominantly during `off-line'
    resting behaviors such as drinking, feeding, grooming, whisking, or being
    still; but can also be seen during the `exploratory' theta-dominated states
    \citep{Oneill2006}. During sleep, SWRs are observed in the non rapid eye
    movement (nREM) phase of the sleep, the slow-wave sleep (SWS). The SWR
    events are associated with a massive population bursts of activity where
    large numbers of cells are synchronously active. The fraction of engaged
    cells can vary from 1\% to 40\% but a typical number is 10\%
    \citep{Mizuseki2013}. SWRs co-occur with the thalamo-cortical spindles. It
    is hypothesised that the this co-activation facilitates the communication
    between the hippocampus and the neocortex \citep{Sirota2003}.
    
    While the exact functions and mechanisms for the generation of SWRs are
    still illusive, it is known that they are involved in the formation and
    consolidation of new memories. According to the dominant theory in the
    field, i.e., the two-stage memory hypothesis \citep{Buzsaki1989}, SWRs
    perform information transfer of memory traces that are temporally stored in
    the hippocampus to the distributed neocortical network. This hypothesis is
    supported by a few lines of experimental observations. On the one hand,
    higher incidence of spindles and SWRs is observed after learning
    \citep{Eschenko2006, Eschenko2008, Girardeau2014}, and the number of
    successfully recalled items in a learning task is correlated to the
    incidence of SWRs measured in the rhinal cortex
    \citep{Axmacher2008}. Moreover, boosting slow-wave oscillations through
    transcranial stimulation during sleep potentiates memory performance in
    humans \citep{Marshall2006}, and presenting odor cues during SWS, but not
    during REM sleep or wakefulness enhance declarative memories that were
    formed during the odor exposure \citep{Rasch2007}. On the other hand,
    disruption of SWRs during sleep or wakefulness impairs the performance of
    rats on hippocampus-dependent spatial tasks \citep{Girardeau2009,
    Jadhav2012}, presumably deteriorating the memory consolidation process and
    the planning of future actions, respectively.

    Another line of observation supporting the hypothesis that the hippocampus
    acts as a temporal storage of memory traces is the replay of behavior
    sequences that takes place during SWRs. \cite{Wilson1994} showed that
    in rats, hippocampal neurons with overlapping place fields that were
    coactive in the awake state showed strong activity correlations during
    SWRs. Later, \cite{Lee2002} demonstrated that the sequence of experienced
    events is preserved during the SWR replay, but temporally compressed $\sim
    20-\rm fold$. More recent experiments reveal that the hippocampal SWR
    replay is coordinated with a grid cell replay of sequences in the
    entorhinal cortex \citep{Olafsdottir2016}. Moreover, \cite{Oneill2017}
    demonstrated that replays in superficial layers of the EC can occur
    independently of the hippocampal sharp waves. In that last study, however,
    the EC replays followed synchronous CA1 population events suggesting that
    the replays can be nevertheless triggered from the hippocampus.

    An amazing amount of data supports the memory-consolidation hypothesis of
    sharp-wave ripples, but however, some experimental results have challenged
    the model. \cite{Foster2006} showed that immediately after a spatial
    experience, the most recent spatial episodes are replayed in a temporally
    reversed order. While such reverse replays are assumed to consolidate the
    most recent experiences, their appearance questions the relevance the
    classical spike-timing dependent plasticity (STDP) rule with an asymmetric
    window \citep{Bi1998} as a mechanism for sequence storage. A somewhat
    bigger challenge to the classical model have been the reports of preplay of
    behaviour sequences. \cite{Dragoi2011} measured the hippocampal neuronal activity
    during SWRs prior to animal's exposure in a novel environment and reported
    replay of place-cell sequences representing the not-yet exposed
    environment. Such results suggest that sequences are not rapidly stored in
    the CA3 region, but that they might be precoded. An estimation based on
    multiunit activity recordings predicts that the rat hippocampus might store
    sequences of about 15 different novel environments \citep{Dragoi2013}.
    
  \subsection{Generation mechanisms}
    Sharp-waves ripples can be observed in the whole hippocampal region. They
    are most prominent in the CA regions revealing a stereotypical laminar
    profile signature in the local-field potential (LFP) (see
    Figure~\ref{fig:swr}B). A current-source density analysis shows that sharp
    waves (SWs) are associated with a large negative deflection of the LFP
    (sink) in the stratum radiatum, which reflects the excitatory input
    impinging on the apical dendrites of the pyramidal neurons. The ripple
    component is most prominent in the stratum pyramidale where the local
    potential shows a fast oscillation ($\sim200\rm \, Hz$) riding on the top
    of a positive deflection.

    The temporal order of local population activations suggests that the origin
    of the SW burst is in the CA3 region, presumably initiated by a build-up of
    recurrent activity \citep{delaPrida2006, Ellender2010, Schlingloff2014,
    Hulse2016}, and then the activity propagates to the CA1 region through the
    Schaffer collateral projections \citep{Csicsvari2000}. The propagation of
    activity from CA3 to CA1 is important for memory formation. Blocking the
    excitatory CA3-CA1 projections impairs the formation of
    hippocampus-dependent memories in mice \citep{Nakashiba2008}.
    Interestingly, SWRs were still observable in the CA1 region, although with
    a lower incidence, ripple frequency, and number of ripples per event
    \citep{Nakashiba2009}. This result suggests that the CA1 region is also
    able to generate sharp waves independently of CA3, when integrating inputs
    from cortical and subcortical areas.
    % of activated pyramidal neurons: from 1 to 50% 

    The ripple oscillation is amongst the most synchronous activity patterns in
    the mammalian brain. A few classes of models have aimed to explain the
    origin of the fast oscillations. One model class relies on the electrical
    coupling between axons of pyramidal cells in the CA3/CA1 regions
    \citep{Draguhn1998, Schmitz2001, Traub2012, Vladimirov2013}. In another model
    class the supralinear amplification of precisely synchronised inputs
    \citep{Memmesheimer2010, Jahnke2015} leads to discrete waves of activity
    that generate ripple oscillations while propagating. The third proposed
    mechanism relies on networks of inhibitory neurons. According to the model,
    sparsely connected networks of fast inhibitory neurons can enter sparsely
    synchronous regime, where the population oscillation can reach frequencies
    of $\sim 200\, \rm Hz$ \citep{Brunel2003}. While the proposed mechanisms are
    not mutually exclusive, experimental results are in favour
    of the inhibition-generated ripples \citep{Buhl2005, Schlingloff2014,
    Donoso2017}.

    \textit{In-vivo} data has suggested that SWRs are local hippocampal events.
    For example, decoupling the hippocampus from the rest of the brain by
    removing the main inputs led to spontaneous dynamics largely dominated by
    SWRs \citep{Buzsaki1983}, pointing out that SWRs might be a `default'
    network state for the hippocampus. \textit{In-vitro} slice models have been
    developed to facilitate the study SWRs \cite[e.g.,][]{Maier2002, Maier2003,
    Kubota2003, Colgin2004}. Horizontal slices of thickness around 300--$600\,
    \mu m$ from the ventral hippocampus are typically used for \textit{in-vitro}
    studies. The slice models offer clear advantages in investigating the
    synaptic, the cellular, and the network properties of the neuronal circuit
    by giving more control to the experimentators. SWRs \textit{in vitro} show
    preserved main characteristics such as spontaneously generated events,
    propagation pathway, laminar profile of the LFP, ripple oscillations,
    neuron participation, and synaptic inputs. Moreover, SWRs can occur with
    conserved waveforms over long periods of time (in the order of hours), and
    single PC firing is coupled to certain waveforms but not to others.
    \citep{Reichinnek2010}. There are, however, also some differences: SWRs
    \textit{in vitro} tend to be shorter and to have a slightly higher ripple
    frequency (150--$180 \,\rm Hz$ \textit{in vivo}; $\sim 200$\,\rm Hz \textit{in
    vitro}). It is worth mentioning that even \textit{in-vivo} SWRs show
    slightly different properties depending on the behaviour state and even on
    the location of the animal \citep{Buzsaki2015}.
    %cite chapter from R and N? on more detailed comparison
    
    Regardless of the focused work of many scientific groups, the mechanisms
    behind the SWR initiation are still not well understood. As SWRs are large
    population events, traditionally it is assumed that they originate from the
    build-up of excitatory activity mainly in the CA3 region
    \citep{delaPrida2006, Ellender2010, Schlingloff2014, Hulse2016}. For
    example, the excitation of a large number of cells in the local circuit by
    an external stimulation can lead to SWRs events \citep{Behrens2005,
    Nimmrich2005, Both2008}. Recent evidences, however, point out to the
    involvement of the interneuron network into the initial phases of the
    events \citep{Sasaki2014, Bazelot2016}. \cite{Schlingloff2014} and
    \cite{Kohus2016} have shown that activation of parvalbumin-positive basket
    cells in slices can lead to SWR events within milliseconds. SWRs can occur
    also by inducing a single action potential of a CA3 pyramidal neuron.
    \cite{Bazelot2016} show that the induced spike of a PC was followed by
    putative interneuron firing within a small delay $<3\,\rm ms$ indicating
    the relevance of the interneurons in SW generation.

    %ca1 mini slice
    %- involvement of cells and theories on their origin
    %- number of cells, results in various sw size, subjective threshold, influencing results

    Other open questions are what does terminate the SWR event, and what
    mechanisms control the incidence of their occurrence. In the
    Chapter~\ref{chap:swr} of this thesis I tackle these problems. There, I
    provide some further literature overview on how SWRs incidence is
    controlled in experimental settings through pharmacology and other means,
    and propose a phenomenological model about sharp-wave ripple generation
    based on the latest experimental results from the literature.

\section{Scope}
  The focus of this thesis is on the mechanisms of SWR generation and the
  associated sequences replay in the hippocampus. In Chapter~\ref{chap:ass}, I
  present a computational model of assembly sequences that aims to explain the
  replay during SWRs. There, I explore in what conditions assembly sequences
  can be replayed by external inputs, or manifest themselves as activity
  patterns that emerge spontaneously. In Chapter~\ref{chap:swr}, I review
  experimental findings hinting at the origins of the sharp-wave complexes,
  analyse data from \textit{in-vitro} recordings, and test hypotheses in
  analytical and \textit{in-silico} models. There, I discuss in-detail a
  minimal hypothetical model that can give rise to the sharp waves observed
  \textit{in vitro}. 
  
\section{tao}
  - SWRs: both2008, csicsvari99009900...


