\chapter{Introduction}

memory, sequences, hippocampus...

\section{Sequences as behaviour substrate: a historical overview}

  \textit{ There is a story that Simonides was dining at the house of a wealthy
    nobleman named Scopas at Crannon in Thessaly, and chanted a lyric poem
    which he had composed in honor of his host, in which he followed the custom
    of the poets by including for decorative purposes a long passage referring
    to Castor and Pollux; whereupon Scopas with excessive meanness told him he
    would pay him half the fee agreed on for the poem, and if he liked he might
    apply for the balance to his sons of Tyndaraus, as they had gone halves in
  the panegyric.  }

  \textit{ The story runs that a little later a message was brought to
    Simonides to go outside, as two young men were standing at the door who
    earnestly requested him to come out; so he rose from his seat and went out,
    and could not see anybody; but in the interval of his absence the roof of
    the hall where Scopas was giving the banquet fell in, crushing Scopas
    himself and his relations underneath the ruins and killing them; and when
    their friends wanted to bury them but were altogether unable to know  them
    apart as they had been completely crushed, the story goes that Simonides
    was enabled by his  recollection of the place in which each of them had
    been reclining at table to identify them for separate interment; and that
    this circumstance suggested to him the \textbf{discovery of the truth that
    the best aid to clearness of memory consists in orderly arrangement}.  }

  \textit{ He inferred that persons desiring \textbf{to train this faculty must
    select localities and form mental images of the facts they wish to
    remember and store those images in the localities, with the result that
    the arrangement of the localities will preserve the order of the facts,
    and the images of the facts will designate the facts themselves}, and we
    shall employ the localities and images respectively as a wax writing tablet
    and the letters written on it.}\footnote{Cicero. On the Orator: Book 2,
    86.352--54. Translated by EW Sutton, H Rackham. Cambridge, MA: Harvard
    University Press, 1967.}

  Simonides of Ceos was a celebrated lyric poet in Ancient Greek world that
  lived in the period 556--468 BC. He has the fame of the inventor of 4 letters
  of the Greek alphabet ($\eta, \, \psi, \, \xi,\,{\rm and } \,\, \omega$) and
  is considered as the first commercial poet who created songs and odes for pay.
  Moreover, he is in the focus of the oldest reference (from 80 BC;
  \ref{Rhetorica}) for a mnemonic technique called `Method of Loci' or `Memory
  palace' (see the legend above from \ref{Cicero}). The method of loci has
  survived over the centuries as a technique to help orators, artists, clerks
  and encyclopedist to remember information of virtually any quality and
  quantity \ref{Yates66}. Nowadays, the method takes a central place in the
  training of every memory champion \ref{Foer2012}.
  
  How does this mnemonic technique work? Can a naive person actually remember a
  large list of items consisting of tens or even hundreds of elements? In its
  essence, the method of loci is a mnemonic technique based on images and
  places. To deploy the method, one needs a predefined trajectory in some very
  well known spatial environment (e.g., a good starting point for beginners is
  imagining home). Then, to imprint the list, each item is sequentially placed
  in the virtual trajectory (e.g., entrance door, telephone cupboard,
  bookshelf, door to the living room, sofa, sofa table, etc.) and imagined as
  vividly as possible. It is suggested to recruit as many sensory modalities as
  possible when imagining the placed objects as this creates more hooks for
  later revoke of the items. For the following recall one needs only to do a
  virtual walk through the same trajectory and to recollect the items from
  their places (loci). 

  %Laura's subject with super memory, having a condition of astasia? has been implementing a similar method, probably unconsciously.. (or a paragraph down?)
  %every super memory human has been using some memory technique that relies on methos of loci or story
  %creation. Even Luria's subject...colorful memory
  %sequences, tables that could be read by raws, columns and diagonals; after 16 years still remembers
  %using streets from his home town or moscow to place items that has be remembered when given very long lists
  %very strong imagination,  severe synestisia where he feels the color, shape, weight, touch, etc of every word he hears or reads..certain problems in concntration while reading and especially while eating..
  %He never changed items with similar items or with synonims, however, during recall often he could skip some item that was placed in a `dark place' or was not very well distinguished form the background..
  %weak in the logical `organisation of memorizing'
  %bad memory on faces: they are so changeable, changing colors all the time, depending on the mood of the person, on the moment of meet..

  %It is, however, an open question whether the sequential arrangement is an
  %universal quality, or whether some more complex schemata can be used for
  %memory formation and recall. For example, can a two-dimensional pattern be
  %used as a base for the storage of new memories?

  Since ancient times it known that memory can greatly benefit from predefined
  sequences. All known mnemonic techniques such as the method of loci, the
  memory palace, and the story method rely on a sequential storage and recall
  of memories.

  Moreover, sequences have been proposed also as a more general model for the
  occurrence and development of the mental processes. The oldest known
  references to the concept of association of thoughts and ideas comes from the
  ancient world \cite{Plato:Phaedo, Aristotle:demomoria}. Later, in the western
  world the thought sequence concept was further developed by philosophers
  \cite{Hobbes}, \cite{Locke}, \cite{Hume} \cite{Stewart}. David Hartley,
  founder of associationist school of psychology in the 18th century proposed
  that even the most complex thought processes can be explain as sequential
  activations of clusters of elementary senses and representations. At the turn
  between 19th and 20th century, a lot of work devoted on the associations was
  contributed in the field of psychology. The thought experiment of William
  James' bear attempts to explain the physiological response of the body as
  sequence of mental events after the sensory arousal of seeing a bear
  \cite{James1854}. In 1897 Pavlov published his seminal work on the
  conditioned reflex \cite{Pavlov}, where he demonstrated the physiological
  effects of classical conditioning. Further work in experimental psychology
  conceived that sequences of experiences and their representations are in the
  base of motor action, cognition and judgment \cite{Watt1904, Titchener1905,
  Washburn1916}.

  In the 1940s, Donald Hebb, a Canadian psychologist and neurophysiologist
  stated that the problem of understanding behavior is the problem of understanding the total action of the nervous system and vice versa.
  He was the first to apply the principles of the neuronal
  doctrine \cite{Cajal} to explain animal and human cognition. In his seminal
  work \cite{Hebb49} he articulated three concepts that are widely used
  nowadays in the neuroscience community. The first idea is known as `Hebbian
  learning': it states that if 2 neurons are connected and fire in close
  temporal proximity, the efficacy of the synapse connecting them should be
  increased. Neurons that fire together form stronger connections and organize
  into assemblies representing abstract mental concepts, now known as `Hebbian
  assemblies'. The third, less known principle is about `phase sequences': once
  an assembly is activated, i.e., the underling neurons are firing, its
  activation would propagate and activate another assembly, thus mental
  concepts would ignite associated concepts. Hebb suggested that such
  sequential activation of assemblies of neurons underlies our most complex
  mental processes. Without going into much details, he suggested that phase
  sequences can interact with each other and organize more complex hierarchies
  and sequence.

  figure from Hebb's book here!

  Since Hebb's work, enormous amount of work has been conducted around the above-mentioned concepts. 


  stdp- markram, pee and poo, INs stdp..gerstner, richard

  assemblies, physiology, fmri?

  phase sequence: some references...

  wikkens, knochblauch, other reviews on assemblies, representations, enigrams, etc...

  get chasing the assembly, trace


  - neuroscience:
    Cajal: neuron doctrine, neurons have 3 parts and propagate impulses only in
    one direction from one neuron to other; bold assumtption from his side
    based solely on anatomical observations... ref textura del sistema nerviso... and linas2003, Nature 


    - from psychology(washburn watson) to neurophysiologistb (lashley, in The problem of serial order in behavior) talking about hierarchical organization of plans, and chunking..

    Ironically, Lashley's student, the founder of CNS propoes PSs..


  - our linear thinking is kind of boring and limiting. is there a way out?
  %%%%%%
  tao:
    - make a structure of this shit above (.5h)
    - write that shit (2h)

\section{sequences in the brain}

  -hebb

  eichenbaum2000: hippocamal activity as indexes to coretical representation of events..episodic sequences
  chasing the trace by Santoro2014; 2 page review

  \subsection{songbirds}
    - review all songbird literature, and a bit of SFCs (1 day)
    Yu96: song sequences: some hierarcical representations..
    Dave2000: song activity replay during sleep
    Hahnloser2002: ultra sparse code k
    Doupe2004: 
    Kosche2015: zebra finch birdsong production relying on patterned inhibition and excitation 
  \subsection{PFC}
    - reverberation (1 day)
    - monkey hand movements

    Kenet2003: spontanously emerging cortical representation of visual atributes
    
    Luczak and maclean2012: spatiotemporal structure...



    contreras2013: reverberation are enchanced during cortical desynchronisation


    -krushkal2013: barrel thalamocortical slices: structured spatiotemporal activity...

    Sadovsky2013: similar topology in different cortical areas, structured temporal activity; precise neuronal firing sequences as well

  \subsection{hippocampus}
    - hippocampus is super rich in oscillations, rhythms and sequences
    - theta, SWRs.. more details is following


    Mauer2012: suggesting that at higher running speeds hippocampal theta seqs generate short-time scale prediction of the future.


  \subsection{elsewhere}
    
  Almeida-Filho: phase sequences in cortex and hippocampus...

    culture:

      sequences in cultured neuronal population Segev2004

      Plenz2002: avalanches can be also temporally precise sequences...

    other animals: 

      Fushiki2016: wave propgataion of muscle contraction in drosdophila (through neurons..); with various patterns/sequences???


      Romano2015: zebrafish larvae: spontaneous assembly activations detected; activity was similar to the sensory driven; showing that spontaneous activity shows some network preference...similar idea in retina, cited by murphy and millier





    maybe in this section to end up that every pattern of activity can be presented as ff propagating activity in an transformed ff network (Goldman2010?)


\section{hippocampus: a brief survey}
  The aim of this section is to give a more detailed description of the
  hippocampus.  I start with a brief historical overview of how our knowledge
  about function of the hippocampus evolved and then dive into a more detailed
  description of its anatomy. 
  \subsection{Hippocampus and memory}
    - history of hp function hypothesis
    - HM
    - plasticity- memory hypothesis 
    - 2-stage memory model
  \subsection{Anatomy}
  \subsection{CA3 microcircuit}
  \subsection{Sharp-wave ripples}

\section{Sharp-wave ripples}
  \subsection{generation}
    - in vitro models
    - inhib-generated, PY-priming, activity build-up

    - swr invitro share number of properties with SWRs invivo: spontanoues generation, propagation from CA3-to-ca1, sharp wave is similar, usually longer lasting in vivo, and also the ripples tend to be faster in vitro
  \subsection{drug influence on SWRs incidence}
  
  GB: 
    The role of $\rm GABA_B Rs$ in the modulation of SWRs is unclear. Various
    studies have reported that blocking the $\rm GABA_B Rs$ in slices doesn't
    alter the SWR properties, such as incidence, amplitude, duration
    \cite{Hollnagel2014, Hofer2015}
    It has been shown that $\rm GABA_B Rs$ are involved in SW modulation 
    \cite{Maier2102}
    but also Behrens? showed that in their model it did not modulate incidence?



    - lit overview
    temperature
    - temp increases incidence, at higher temps, there is a bifurcation and swrs come in bursts (papa...)

pH
  - ???

NNC-711 (Viereckel2013)

thiopental (Papatheodoropoulos2007, Wittinggton96, Dickinson2002)
  - increases IPSPs (Dick, popa)
  - a sedative agent impairing memory, barbiturates
  - decreases SW incidence in vitro (papa...)
  - prolongs single SWs (papa..)
  - possible explanations: reduces excitability, enhances tonic gAR on PCs and INs () and directly activates gAR

propafol
  - tonic gAR, similar results to thiopental (Bieda2004)

phenobarbital
  - also a barbiturate
  - at low concentration (up to 100uM) it increases SW incidence; at higher concetration it decreases incidence
  - increases amplitude 

diazepam, zolpidem (Koniaris2011, Wittington96, Pawelzik99, Thomson2000, Zarnowska2009)

gabazine

gB drugs

canabinoids



\section{Scope}


\section{TODO}
check philosophy of sequential replay

West91: CA1 is 350,000 neurons

Rapp96: CA3: 240,000

Bezaire2013: review on numbers

Cai2016: A shared neural ensemble links distinct contextual memories encoded close in time

Luongo2016:Correlations between prefrontal neurons form a small-world network that optimizes the generation of multineuron sequences of activity 

Mochizuki: similiraties and diffs in brain region firing porperties across species

shein-idelson2016: SW in dragons



refs:
  %Luria, small book for big memory


  Plato, Phaedo

  Aristotle, De Memoria et Reminiscentia, trans. H. Reid.

  Frances A. Yates, Art of memory, 1966

  Ad C. Herennium de ratione dicendi (Rhetorica ad Herennium), book 3, unknown author, 80 BC, https://archive.org/details/adcherenniumdera00capluoft

  Cicero. On the Orator: Book 2, 86.352--54. Translated by EW Sutton, H Rackham. Cambridge, MA: Harvard University Press, 1967.}

  Foer2012, Moonwalking with Einstein

  "Hobbes's Moral and Political Philosophy". Stanford Encyclopedia of Philosophy. Retrieved 11 March 2009.
  Locke, An Essay Concerning Humane Understanding, Volume I.
  Hume, A Treatise of Human Nature
  Hume, An Enquiry concerning Human Understanding
  David Hartley, Observations on Man, His Frame, His Duty, and His Expectations (2 vol., 1749) 
  Stewart, D. (1855). Philosophical essays. In W. Hamilton (Ed.), The collected works of Dugald Stewart (Vol. V)Edinburgh: Thomas Constable and Co.
  james, "What is an Emotion?" Mind, vol. 9, 1884, pp. 188-205
  Watt, HJ. Experimentelle Beiträge zu einer Theorie des Denkens. W. Engelmann, 1904.
    sequence of experiences lead to judgement
  Edward B. Titchener's Experimental Psychology (1905)
  Washburn, M. F. (1916)Movement and Mental Imagery: Outlines of a Motor Theory of the Complexer Mental Processes. Boston, MA: Houghton Mifflin Company.



