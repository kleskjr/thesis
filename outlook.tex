\chapter{Outlook}
%summary
  The aim of this final chapter is to put into perspective the problems that
  have been analysed so far. In particular, what is the relevance of the
  assembly sequence in the framework of the disinhibitory 3-population model?
  Moreover, I openly discuss some ideas that have spontaneously occurred over
  the course of the last few years.
  %in SWRs, of the real replay, preplay..
  %if stuff gets binded to predifined seqs, why and how?

\section{Bridging the assembly sequence model and the 3-population hypothesis}
  Neural assemblies connected in a feedforward manner can explain the replay of
  behaviour sequences observed in the hippocampus. The sequence replay can be
  evoked by an external input cue, or occur spontaneously. As described in
  Chapter~\ref{chap:asss}, a spontaneous replay requires higher connectivities
  between assemblies than the evoked replay. Moreover, the spontaneous replay
  largely depends on the network excitability, and a small change of the net
  input can shift to or from spontaneous replay regime. For example, decreasing
  the total input to the network abolishes spontaneous events, while an
  increase of the input facilitates them. However, such circuitry relying on
  two populations (an excitatory and an inhibitory one) cannot account for the
  inhibitory-evoked SWRs observed \textit{in vitro}
  \citep[e.g.,][]{Schlingloff2014, Kohus2016}. Therefore, in
  Chapter~\ref{chap:swr}, I described a phenomenological model of a hippocampal
  network consisting of 3 populations (one excitatory and two inhibitory
  populations). The underlying idea is that the activation of one inhibitory
  population activates a disinhibitory circuit, leading to a population burst.
  An ultimate goal, that has not been achieved in this thesis, is to combine
  these two models (e.g., the assembly sequence and the 3-population model)
  into a single framework. Is the high--firing--rate regime during a population
  burst going to dramatically decrease the connectivities required for a
  replay?

  A possible limitation of the presented assembly-sequence model is the fact
  that it relies on `rigid' assemblies of neurons that are uniformly connected
  among themselves. Experimental data on the synaptic organisation in
  hippocampus, as well as in the cortex, show heavily skewed distributions of
  synaptic strength and connectivity \citep[e.g.,][]{Takahashi2010,
  Buzsaki2014, Cossell2015}. Lognormal distributions of connectivities and
  synaptic weights between neurons would be a more plausible model for a cell
  assembly. Then, a few `central' neurons will fire reliably during assembly
  activation, while the participation of the `periphery' neurons will depend
  more on the strength and the type of input stimulation. To what degree such
  modification of the model will affect the results of sequence replay is not
  known.
  %different neurons can join the neurons depending on the context, the modalities, and the evoked depth of the concept

\section{Further implications of the disinhibitory circuit}
  %here theta: theta seqs...?
  The 3-population hypothesis is a plausible general mechanism for achieving
  multistability of local networks. Particularly, a disinhibitory pathway can
  lead to a rapid activation of the excitatory network upon command. Such a
  mechanism might underlie the gating of spontaneous activity not only in the
  hippocampus, but also in local cortical networks. \cite{Luczak2013} suggested
  that the sensory cortex is constantly processing the inputs in a background,
  low-firing rate regime, i.e., the `down state'. Upon a top-down request from
  higher-order areas, the sensory cortex switches to an `up state', and thus,
  gates the sensory information upstream. Such up states might be achieved
  through a rapid disinhibition.

  A line of research from Brecht lab points out to the plausibility of the
  disinhibitory hypothesis in cortical networks. \cite{Brecht2004} has shown
  that a repetitive stimulation of a single pyramidal neurons in the rat motor
  cortex can evoke whisker movement. The stimulation evoked a marked change of
  pattern activity reflected in the field potential and increased firing of the
  fast spiking, putative inhibitory interneurons \citep{Houweling2007,
  Doron2014}. Moreover, stimulation of fast spiking, putative inhibitory
  interneurons in somatosensory cortex led to stronger effects detected in the
  animal's behaviour compared to a stimulation of principal cells
  \citep{Doron2014}. The stimulations were typically followed by a
  `long-lasting' inhibition of firing rates measured in the microcircuit
  \citep{Doron2014}. Experiments targeting the disinhibitory neurons would
  address explicitly the question of disinhibition in cortical networks.
  
  %in hp, similar effects can be achieved also with theta: theta seqs are at particular phase of
  %theta, when there is more excitation; are theta seqs intrinsic, maybe not.. not that small time delays
  %what is the role the oscillation gamma, ripple?
  %givin a particular window of opportunity, when the inputs are also optimized for that window; no need
  %to integrate over long intervals, which is not possible
  %ripple discretizes the float of information
  %discrete assemblies are more efficient ? or there is always some level of interplay between assemblies?
  %embedded assemblies? or multiple assemblies that integrate inputs and act like neurons; in the case of ripples they are more prominent

  %participation or does it only give windows of opportunity for firing?
  %information on the order of PC firing can be transfered also through the inhib population.
  %giving certain times during the ripple windows...(ref on inhib propagation of activity norman?)

  %% role of ripples
  %norman2006: algorithm relying on inhibitory learning that facilitates weak memories
  %and punishes the activation of competitor memories

\section{Why a hippocampal replay?}
  According to the standard 2-stage memory model, behaviour sequences are
  imprinted in the CA3 recurrent networks upon experience \citep{Marr1971,
  Buzsaki1989}. However, such view has been challenged in the last years as
  reports have shown that the hippocampal replay is not a simple function of
  experience \citep[e.g.,][]{Gupta2010}. Even more explicitly,
  \cite{Dragoi2011, Dragoi2013} showed that sequences are replayed in the
  hippocampus already prior to the first exposure of the environment which
  these sequences represent. What could be the role of the observed replays? 
  
  Dragoi suggested\footnote{The episodic event took place during a public talk
  at BCCN-Berlin in 2013.} that the role of CA3 is to create blank sequences
  that are mapped later to experience. \cite{ChengS2013} proposed the CRISP
  model which goes along these lines. According to the model, CA3 sequences are
  formed offline prior to utilization due to the maturation of newly generated
  granule cells in the dentate gyrus. During the first exposure to the
  environment, granule cells that carry sparse information about the
  environment are instantly mapped to the CA3 sequences via the mossy fibers.
  While this is a plausible explanation of how the hippocampal input is mapped
  to the blank hippocampal sequences, it is not clear how the hippocampal output
  during a following offline replay is mapped back to the context it
  represents. Such mapping would require highly plastic synapses that are able
  to encode information in a single experience. \cite{ChengS2013} hypothesises
  that the internal sequences are mapped back to the behavioural information
  via the feedforward synapses from CA1 to the deep EC layers. Such learning
  might be facilitated by the preplays that occur prior to experience. In this
  way, the internal CA3 sequences would not be purely blank, but are already
  integrated with activity patterns in the deep layers of the EC.

  %EC replay inde[endent of CA1?


  %clear up the rest

  %one option is that sequences are there, but elements are attached to the seq,
  %this will require much less creation of new synapses and assemblies.
  %scale-free assembly with new members joining, then each further replay will
  %consolidate the new elements into the assembly...
  

  %Many questions remain unanswered to date about the origin and function of the
  %hippocampal replay. If sequence replays are intrinsic property of the CA3
  %region, then two links have to be created on demand upon learning: context
  %--> CA3 sequence; and CA3 sequence --> context.  

  %However, in vivo SWR replays
  %might be modulated by external inputs, from EC, or subiculum..  If external
  %information drives the replay, then there is no need for rapid hp output
  %mapping..?

  %Another possible link between the input and the output of the hippocampus can
  %also happen in the EC, on the synapses between the superficial and deep
  %layers.


%here in the last pages, I will track a hypothetical experiment (geist experiment???)
%jennifer and oma..

%jenifer and oma are heavily connected; in the context of neuroscience, one concept is
%likely to evoke the other ; outside of the context, probably not as much connected
  %connection between abstract concepts, fast and to the point
  %- our linear thinking is kind of boring and limiting. is there a way out?
  %conceptualized knowledge is linear; thinking relies on the distributed net of concepts that can bring surprises;
  %once we live in the hp, however, it's linear
  %implicit knowledge is non-linear
  %we linearize and consolidate patches of linearized associations...the resulting network of concepts might be quite complez and of different quality as presented in hp
%

  %distributed...place cells -> grids

%propse experiment: find the cheese neuron in DG and activate it each time animal at a place
%can work with any object neuron, and then associate the object with cheese, and let the animal in that env with virtual object again :)

  %probably in hippocampus can become as sparse and sharp as they can. Further downstream (upstreams) assemblies are getting blurrier and more mixed.

  %sparse info at the input, then smears out in the cortex but is also sparse in the hp. that's how we create declarative representations or memories of abstract ideas that do are not necessary present to the brain by the sensory system; a place, or a word describing the tree we see. these 
  %- the input is not sparse actually, as uniform as it gets

  %assembly sequences in simpler neural systems might be more prevailing?

  %can the sparseness have a few dimensions, what I mean is that from orientation-maps the further levels of visual processing go to more complex representations that in a way are sparser? are they?

  %jennifer aniston neuron..what is the advantage of such coding
  %for making a link between abstract concepts (aniston and our grandma), maybe it's good to make this link 
  %at the highest abstraction, rather than at the level of subabstractions..

  %while my grandma is associated with the house that I used to visit her at,
  %with the stories that she used to tell me, or more generally with any memory and experience
  %that has numereous emotianal attachments to it and that shapes me as a self.
  %jeniffer aniston is connected mostly with some TV show, such as friends or
  %other cheesy romantic movie, and possibly with remote memories while seening
  %these shows.
%
  %so what happens when i connect my grandma and aniston in a typical mnemonic
  %task?

  %while revoking their representations, multiple areas in the cortex are boiling
  %by igniting assemblies that constitute the mental concepts. These assemblies
  %can be distributed over the whole cortex and represent various concepts such
  %as events, people, emotions, etc. To create links between this diverse zoo of
  %concepts should be a tricky task, prune to errors due to noise.. However, in
  %the hippocampus, the representations are on the highest level: between oma
  %and aniston; the question is how this connection `backpropagates' to the
  %founding concepts. While we don't know the answer to this question, some relatively
  %recent finding from the EC might give a hint. Grid cells in EC, major input/output of the hippocampus.


