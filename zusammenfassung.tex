\huge\textbf{Zusammenfassung}
\normalsize
\vspace{10mm}

Komplexe Muster neuronaler Aktivit\"{a}t entstehen w\"{a}hrend der Sharp-wave
Ripples (SWRs) im Hippocampus und w\"{a}hrend der Up States im Neokortex
(Zust\"{a}nden mit hoher Aktivit\"{a}t). Sequenzen von Verhalten, die in der
Vergangenheit erlebt wurden, werden w\"{a}hrend des komplexen Musters
abgespielt. Die zugrunde liegenden Mechanismen sind nicht gr\"{u}ndlich
erforscht: Wie k\"{o}nnen kleine synaptische Ver\"{a}nderungen die
gro{\ss}fl\"{a}chige Netzwerkaktivit\"{a}t w\"{a}hrend des
Ged\"{a}chtnisabrufes und der Ged\"{a}chtniskonsolidierung kontrollieren?

Im ersten Teil dieser Abhandlung wird die Hypothese aufgestellt, dass eine
schwache synaptische Konnektivit\"{a}t zwischen Hebbschen Assemblies von der
bereits vorhandenen rekurrenten Konnektivit\"{a}t gef\"{o}rdert wird. Diese
Hypothese wird auf folgende Weise gepr\"{u}ft: die vorw\"{a}rts gekoppelten
Assembly-Sequenzen werden in neuronale Netzwerke eingebettet, mit einem
Gleichgewicht zwischen exzitatorischer und inhibitorischer Aktivit\"{a}t.
Simulationen und analytische Berechnungen haben gezeigt, dass rekurrente
Verbindungen innerhalb der Assemblies zu einer schnelleren
Signalverst\"{a}rkung f\"{u}hren, was eine Reduktion der notwendigen
Verbindungen zwischen den Assemblies zur Folge hat. Diese Aktivit\"{a}t kann
entweder von kleinen sensorisch \"{a}hnlichen Inputs hervorgerufen werden oder
entsteht spontan infolge von Aktivit\"{a}tsschwankungen. Globale --
m\"{o}glicherweise neuromodulatorische -- \"{a}nderungen der neuronalen
Erregbarkeit k\"{o}nnen daher die Netzwerkzust\"{a}nde steuern, die
Ged\"{a}chnisabruf und die Konsolidierung beg\"{u}nstigen.

Der zweite Teil der Arbeit geht der Herkunft der SWRs nach, die in vitro
beobachtet wurden. Neueste Studien haben gezeigt, dass SWR-\"{a}hnliche
Erscheinungen durch optogenetische Stimulation der Subpopulationen von
inhibitorischen Neuronen hervorgerufen werden k\"{o}nnen
\citep{Schlingloff2014, Kohus2016}. Um diese Ergebnisse zu erkl\"{a}ren wird
ein de-inhibierendes Schaltkreis-Modell diskutiert, das die beobachteten
Populationsausbr\"{u}che generieren kann. Die Auswirkungen der
pharmakologischen GABAergischen Modulatoren auf die SWR-H\"{a}ufigkeit werden
in vitro untersucht. Die gewonnenen Ergebnisse wurden in Rahmen des
Schaltkreis-Modells analysiert. Insbesondere wird den folgenden Fragen
nachgegangen: Wie unterdr\"{u}ckt Gabazine, ein GABA_A-Rezeptor-Antagonist, die
Entwicklung von SWRs? Wird das Zeitintervall zwischen SWRs durch die Dynamik
der GABA_B Rezeptoren moduliert?

\newpage
