\huge\textbf{Zusammenfassung}
\normalsize
\vspace{10mm}

Komplexe Muster neuraler Aktivit\"{a}t tauchen w\"{a}hrend der
Zust\"{a}nde mit hoher Aktivit\"{a}t (up states) im Neokortex und sharp-waves ripples
(SWRs) im Hippocampus auf. Dies schlie{\ss}t Sequenzen ein, die solchen Sequenzen
\"{a}hneln, die den Verhaltenserfahrungen vorausgehen. Die zugrunde liegenden Meschanismen
sind noch nicht ganz erforscht. Wie
k\"{o}nnen kleine synaptische Spuren, die auf Erfahrung basieren,
gro{\ss}fl\"{a}chige Netzwerkaktivit\"{a}t w\"{a}hrend Ged\"{a}chtnisabrufen
und Ged\"{a}chtniskonsolidierung kontrollieren?

Im ersten Teil dieser Abhandlung wird die Hypothese aufgestellt, dass
schwache synaptische Konnektivit\"{a}t zwischen Hebb’schen
Assemblies von vorher vorhandenen rekurrenten Zusammenh\"{a}ngen zwischen
diesen gef\"{o}rdert ist. Diese Hypothese wird auf folgender Weise gepr\"{u}ft:
die vorw\"{a}rts gekoppelten Assemblies-Sequenzen werden in zuf\"{a}lligen
neuralen Netzwerken ingebettet, mit einem Gleichgewicht zwischen Exzitation und
Inhibition. Simulationen und analytische Berechnungen haben gezeigt, dass
rekurrente Verbindungen zwischen den Assemblies zu einer schnelleren
Signalverst\"{a}rkung f\"{u}hren, was eine Reduktion der ben\"{o}tigten
internen Verbindungen zwischen den Assemblies zur Folge hat. Diese Aktivit\"{a}t
kann entweder von kleinen sensorisch-\"{a}hnlichen Inputs
hervorgerufen werden oder entsteht spontan infolge von Aktivit\"{a}tsschwankungen.
Globale -- m\"{o}glicherweise
neuromodulatorische -- \"{A}nderungen der neuronalen
Erregbarkeit k\"{o}nnen zwischen Netzwerkzust\"{a}nden umschalten, die jeweils
Abrufen und Konsolidierung beg\"{u}nstigen.

Der zweite Teil der Arbeit geht der Herkunft der SWRs nach, die in in-vitro
Vorlagen beobachtet wurden. Neueste Studien haben gezeigt, dass
SWR-\"{a}hnliche Erscheinungen durch optogenetische Anregung der
Subpopulationen von hemmenden Neuronen hervorgerufen werden k\"{o}nnen
\citep{Schlingloff2014, Kohus2016}. Um diese Ergebnisse zu erkl\"{a}ren wird
ein 3-Populationsmodell diskutiert, welches als de-inhibierender Schaltkreis
angenommen wird, der die beobachteten Populationsausbrüche generieren kann. Die
Auswirkungen der pharmakologischen GABAergischen Modulatoren auf die
SWR-H\"{a}ufigkeit in vitro werden untersucht. Die gewonnenen Erkenntnisse
werden angesichts des vorgeschlagenen enthemmenden Kreislaufs diskutiert.
Insbesondere wird den folgenden Fragen nachgegangen: Wie unterdr\"{u}ckt der
Gabazine, $\rm GABA_A$-Rezeptor Antagonist die Bildung von SWRs? Wird der
Zeitma{\ss}stab der Zwischenvorkomnissabst\"{a}nde (inter-event interval)
durch die verlangsamte Dynamik der $\rm GABA_B$ Rezeptoren moduliert?

\newpage
