\huge\textbf{Zusammenfassung}
\normalsize
\vspace{10mm}

Komplexe Muster von neuralen Aktivitaeten entstehen w\"{a}hrend den SWRs im
Hippocampus und w\"{a}hrend den Up States im Neokortex (Ein Zustand mit hoher
Aktivit\"{a}t). Sequenzen von Verhalten, die in der Vergangenheit erlebt
wurden, werden w\"{a}hrend dem komplexen Muster abgespielt.  Die zugrunde
liegenden Meschanismen sind nicht gründlich erforscht: Wie k\"{o}nnen kleine
synaptische Ver\"{a}nderungen, gro{\ss}fl\"{a}chige Netzwerkaktivit\"{a}t
w\"{a}hrend Ged\"{a}chtnisabrufen und Ged\"{a}chtniskonsolidierung
kontrollieren?

Im ersten Teil dieser Abhandlung wird die Hypothese aufgestellt, dass schwache
synaptische Konnektivit\"{a}t zwischen Hebb'schen Assemblies von der bereits
vorhandenen rekurrenten Konnektivität  gef\"{o}rdert ist. Diese Hypothese wird
auf folgender Weise gepr\"{u}ft: die vorw\"{a}rts gekoppelten
Assemblies-Sequenzen werden in neuralen Netzwerken ingebettet, mit einem
Gleichgewicht zwischen exzitatorische und inhibitorische Aktivit\"{a}ten.
Simulationen und analytische Berechnungen haben gezeigt, dass rekurrente
Verbindungen innerhalb den Assemblies zu einer schnelleren
Signalverst\"{a}rkung f\"{u}hren, was eine Reduktion der notwendigen
Verbindungen zwischen den Assemblies zur Folge hat. Diese Aktivit\"{a}t kann
entweder von kleinen sensorisch-\"{a}hnlichen Inputs hervorgerufen werden oder
entsteht spontan infolge von Aktivit\"{a}tsschwankungen.
Globale---m\"{o}glicherweise neuromodulatorische---\"{A}nderungen der
neuronalen Erregbarkeit k\"{o}nnen daher die Netzwerkzust\"{a}nden steuern, die
das Gedächnisabrufen und die Konsolidierung beg\"{u}nstigen.


Der zweite Teil der Arbeit geht der Herkunft der SWRs nach, die
\textit{in-vitro} beobachtet wurden. Neueste Studien haben gezeigt, dass
SWR-\"{a}hnliche Erscheinungen durch optogenetische Stimulation der
Subpopulationen von inhibitorischen Neuronen hervorgerufen werden k\"{o}nnen
\citep{Schlingloff2014, Kohus2016}. Um diese Ergebnisse zu erkl\"{a}ren wird
ein de-inhibierender Schaltkreis Modell diskutiert, der die beobachteten
Populationsausbr\"{u}che generieren kann. Die Auswirkungen der
pharmakologischen GABAergischen Modulatoren auf die SWR-H\"{a}ufigkeit werden
\textit{in vitro} untersucht. Die gewonnenen Ergebnisse wurden in Rahmen des
Schaltkreis Modells analysiert.  Insbesondere wird den folgenden Fragen
nachgegangen: Wie unterdr\"{u}ckt Gabazine, ein $\rm GABA_A$-Rezeptor
Antagonist, die Entwicklung von SWRs? Wird das Zeitintervall zwischen SWRs
durch die Dynamik der $\rm GABA_B$ Rezeptoren moduliert?

\newpage
