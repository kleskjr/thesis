\huge\textbf{Abstract}
\normalsize
\vspace{10mm}


Complex patterns of neural activity appear during up-states in the neocortex
and sharp-waves ripples (SWRs) in the hippocampus, including sequences that
resemble those during prior behavioral experience. The mechanisms underlying
this replay are not well understood. How can small synaptic footprints engraved
by experience control large-scale network activity during memory retrieval and
consolidation?

In the first part of this thesis, I hypothesise that sparse and weak
synaptic connectivity between Hebbian assemblies are boosted by pre-existing
recurrent connectivity within them. To investigate this idea, sequences of
assemblies connected in a feedforward manner are embedded in random neural
networks with a balance of excitation and inhibition. Simulations and
analytical calculations show that recurrent connections within assemblies allow
for a fast amplification of signals that indeed reduces the required number of
inter-assembly connections. Replay can be evoked by small sensory-like cues or
emerge spontaneously by activity fluctuations. Global---potentially
neuromodulatory---alterations of neuronal excitability can switch between
network states that favor retrieval and consolidation.

The second part of this thesis investigates the origin of the SWRs observed in
\textit{in-vitro} models. Recent studies have demonstrated that SWR-like events
can be evoked after optogenetic stimulation of subpopulations of inhibitory
neurons \citep{Schlingloff2014, Kohus2016}. To explain these results, a
3-population model is discussed as a hypothetical disinhibitory circuit that
could generate the observed population bursts. The effects of pharmacological
GABAergic modulators on the SWR incidence \textit{in vitro} are analysed. The
results are discussed in the light of the proposed disinhibitory circuit. In
particular, how does gabazine, a $\rm GABA_A$ receptor antagonist, suppress the
generation of SWRs? Another explored question is whether the slow dynamics of
$\rm GABA_B$ receptors is modulating the time scale of the inter-event
intervals.

\newpage
