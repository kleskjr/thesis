\chapter{Summary and outlook}

%%%%%%%%%%%%%%
connection with oscillations:
%%%%%%%%%%%%%%
here we assume AI state, however the brain likes to oscillate. Different frequency of oscillations are hypotheisised to give a band for communications between assemblies.[ref on communication through resonance]. 

on fast and slow time scales of learning and plasticity how they match? theta, gama, PP, spwrs

swr interplay with phase precession

ripple discretizes the float of information

discrete assemblies are more efficient ? or there is always some level of interplay between assemblies?
embedded assemblies? or multiple assemblies that integrate inputs and act like neurons; in the case of ripples they are more prominent

probably in hippocampus can become as sparse and sharp as they can. Further downstream (upstreams) assemblies are getting blurrier and more mixed.

sparse info at the input, then smears out in the cortex but is also sparse in the hp. that's how we create declarative representations or memories of abstract ideas that do are not necessary present to the brain by the sensory system; a place, or a word describing the tree we see. these 

assembly sequences in simpler neural systems might be more prevailing?

can the sparseness have a few dimensions, what I mean is that from orientation-maps the further levels of visual processing go to more complex representations that in a way are sparser? are they?


amazing compression of time, but why hippocampus? for declarative memories, only? look at HM, he was good and happy; decision making was intact, feeling basically everything without the fucking declarative memories. A few example of learned words however show that it's not the hippocampus monopole in learning, although not efficiently. So what else? no time no memory! no memory, no time? 
it's not the memories that define us per se but the memories can nevertheless change the personality on a slower time scale. without new memories time just passes and we are death at some moment. with memories on the other side we live in the float of life and keep updating our representation of this float in a way that is often beneficial for the survival. and just then we die :). time wasn't worthless whatever the outcome though!

of course this is only a theory which is being tested and refined over the last years. we keep looking into cracking it and diving deeper, by 


a big scientific question is if HM mindwandered????


mindwandering affects our memories? why not? interesting question! people imagining an learning certain action can do as IF THEY Have done the ac


